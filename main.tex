\documentclass[conference]{IEEEtran}
\IEEEoverridecommandlockouts
\renewcommand\IEEEkeywordsname{Keywords}
% The preceding line is only needed to identify funding in the first footnote. If that is unneeded, please comment it out.
\usepackage{cite}
\usepackage{amsmath,amssymb,amsfonts}
\usepackage{algorithmic}
\usepackage{graphicx}
\usepackage{textcomp}
\usepackage{xcolor}
%\usepackage[table]{xcolor}
\usepackage{colortbl}
\usepackage{arydshln} % Paquete para líneas punteadas y sutiles
\usepackage{hyperref}
\usepackage{caption}
\usepackage{stfloats}  % Permite la figura* en el entorno de dos columnas

%% instaldas nuevas **
\usepackage{makecell} % Importa el paquete makecell
\usepackage{tabularx}
\usepackage{graphicx}
\usepackage{tikz}
\usepackage{colortbl} % Para \rowcolor
\usepackage{xcolor}   % Para definir colores personalizados
%\usepackage{subfigure} 
\usepackage{subcaption}
\usepackage{amsmath}
\usepackage{enumitem}
\usepackage{lineno}
\usepackage{pdfpages}
\usepackage{lscape}
\usepackage{booktabs}
\usepackage{multicol}

\usepackage{hyperref}
% Es recomendable cargar cleveref después de otros paquetes como hyperref para evitar conflictos.

\usepackage{cleveref}
%\usepackage[spanish]{babel} % Para idioma español

%%%%%%%%% ESPAÑOL %%%%%%%%%%%%%%%%%%%
% \usepackage[spanish]{babel}   
% \addto\captionsspanish{\renewcommand{\tablename}{Tabla}}
% % Configuración de nombres en español
% \crefname{section}{sección}{secciones}
% \Crefname{section}{Sección}{Secciones}
% \crefname{figure}{figura}{figuras}
% \crefname{table}{tabla}{tablas}
% %\crefname{equation}{ecuación}{ecuaciones}
% \crefname{Figure}{Figura}{Figuras}
% \crefname{Table}{Tabla}{Tablas}

\captionsetup{labelfont=bf}
\def\BibTeX{{\rm B\kern-.05em{\sc i\kern-.025em b}\kern-.08em
    T\kern-.1667em\lower.7ex\hbox{E}\kern-.125emX}}
\begin{document}

% \title{COMUNICACIÓN HÍBRIDA RADIO-ÓPTICA PARA MEJORAR LOS SISTEMAS DE TELECOMUNICACIONES INDOOR\\
% }
\title{HYBRID RADIO-OPTICAL COMMUNICATION TO IMPROVE INDOOR TELECOMMUNICATIONS SYSTEMS}
\author{\IEEEauthorblockN{1\textsuperscript{st} Juan Navarro-Restrepo}
\IEEEauthorblockA{\small \textit{dept. Electronics and Telecommunications } \\
\small \textit{Institución Universitaria ITM}\\
\small Medellin, Colombia \\
\small \href{https://orcid.org/0000-0002-6031-9940}{https://orcid.org/0000-0002-6031-9940}}
\and
\IEEEauthorblockN{2\textsuperscript{nd} Andrés Betancur-Pérez}
\IEEEauthorblockA{\small \textit{dept. Electronics and Telecommunications } \\
\small \textit{Institución Universitaria ITM}\\
\small Medellin, Colombia \\
\small \href{https://orcid.org/0000-0002-4738-8052}{https://orcid.org/0000-0002-4738-8052}}
\and
\IEEEauthorblockN{3\textsuperscript{rd} Roger Martínez-Ciro}
\IEEEauthorblockA{\small \textit{dept. Electronics and Telecommunications } \\
\small \textit{Institución Universitaria ITM}\\
\small Medellin, Colombia \\
\small \href{https://orcid.org/0000-0001-7592-1742}{https://orcid.org/0000-0001-7592-1742}}
\and
\IEEEauthorblockN{4\textsuperscript{th} Francisco López-Giraldo}
\IEEEauthorblockA{\small \textit{dept. Electronics and Telecommunications } \\
\small \textit{Institución Universitaria ITM}\\
\small Medellin, Colombia \\
\small \href{https://orcid.org/0000-0003-0446-4307}{https://orcid.org/0000-0003-0446-4307}}
\and
}

\maketitle

\begin{abstract}
This paper experimentally proposes a hybrid system that combines Radio Frequency and Visible Light Communication technologies to mitigate signal non-uniformity in indoor environments. Additionally, it presents an experimental study on the degradation of the signal-to-noise ratio of a commercial broadcast signal at the frequency of 103.5 MHz. This analysis was carried out at different levels of the ITM University Institution in Medellín, including basements 1 and 2, and focused exclusively on signal reception.
The results were compared with the initial measurements taken, showing that the pure RF signal, captured conventionally, had an average signal-to-noise ratio of 4.93 dB. After implementing the hybrid system, the average signal-to-noise ratio increased significantly to 36.9 dB. These results demonstrate the effectiveness of the technological convergence proposed in this study to improve signal quality in multiple demanding application scenarios, such as the Internet of Things, Smart Cities, Sustainable Mining, Indoor Positioning, Marine Environment Monitoring, and others.

\end{abstract}
\vspace{3pt}
\begin{IEEEkeywords}
 RF, VLC, SNR, Sombreado, Desvanecimiento profundo, Dispersión, Absorción Atmosferica.
\end{IEEEkeywords}

% \chapter{Introduction} \label{chap:intro}
\section{Introduction} \label{chap:intro}

The deployment of telecommunications networks is increasing at an accelerated pace as communities and territories rapidly grow in population and infrastructure. Currently, there is an exponential increase in traffic on telecommunications networks due to the emergence of services such as teleconferencing, high-definition video streaming, interactive video games, among others. Such services demand a considerable portion of the available bandwidth of communication channels. The need to transmit information has risen to such a point that the various telecommunications network and service providers (PRST) must anticipate growth and readjust to constant changes, complying with the relevant technical specifications and following the regulations established by regulatory bodies such as the Ministry of Information Technology and Communications (MINTIC), the Communications Regulatory Commission (CRC), the National Spectrum Agency (ANE), and other regulatory bodies \cite{MINTIC,CRC,ANE} [1]–[3].
In terms of communication, one of the channels that has the most drawbacks in terms of bandwidth management and use is the wireless channel. In theory, the electromagnetic spectrum is infinite \cite{Heter1985} [4], but only a certain portion of the radio spectrum (see \Cref{tablaA}) is used to transmit information, limiting its use to a few tens of GHz. On the other hand, some of the radio spectrum bands, even if they are not currently in use, must remain free and available for special uses \cite{Hattab2019} [5]. In this sense, to provide commercial communications services, the radio spectrum is further restricted.

\smallskip

\textcolor{blue}{El Despliegue de las redes de Telecomunicaciones se incrementan cada vez más a un ritmo acelerado a  medida que las comunidades o territorios crecen en población e infraestructura rápidamente. Actualmente hay un incremento exponencial del tráfico por las redes de telecomunicaciones debido al surgimiento de servicios como las teleconferencias, streaming de videos en alta definición, videojuegos interactivos, entre otros. Tales servicios demandan una porción considerable del ancho de banda disponible de los canales de comunicación. La necesidad de transmitir información se eleva a tal punto que, los distintos Proveedores de Redes y Servicios de Telecomunicaciones - PRST, deben prever su crecimiento y reajustarse a los cambios constantes, cumpliendo con las especificaciones técnicas del caso y siguiendo las normativas establecidas desde los entes regulatorios como es el caso del Ministerio de las Tecnologías de la Información y las Comuniaciones - MINTIC, Comisión de Regulación de Comunicaciones - CRC, la Agencia nacional del Espectro - ANE  y demás entes normativos \cite{MINTIC,CRC,ANE} .}

\smallskip

\textcolor{blue}{En materia de comunicación, uno de los canales que tiene más inconvenientes en la gestión y uso del ancho de banda, es el canal inalámbrico. En teoría el espectro electromagnético es infinito \cite{Heter1985}, aun así, para transmitir información, se emplea solo una porción determinada del espectro de radio (ver Tabla \ref{tablaA}), lo cual limita su uso hasta unas decenas de GHz. Por otro lado, algunas de las bandas del espectro de radio, aunque no estén usadas en el momento, deben permanecer libres y disponibles para usos especiales \cite{Hattab2019}. En este sentido, para brindar servicios de comunicaciones comerciales, el espectro 
radioeléctrico se restringe aún más. }

 \begin{table}[ht!]
    \caption{Frequency Bands}
    \begin{center}
    \begin{tabular}{cccc}
    \hline
    \textbf{\textit{Band}} & \textbf{\textit{Acronym}}& \textbf{\textit{Frequency}}& \textbf{\textit{Wavelength ($\lambda$) }} \\
    \hline
    4  & VLF & 3 a 30 kHz      & 100 $\sim$ 10km   \\ 
    5  &  LF & 30 a 300 kHz    &  10 $\sim$  1km   \\
    6  &  MF & 300 a 3 000 kHz & 1km $\sim$ 100m   \\
    7  &  HF & 3 a 30 MHz      & 100 $\sim$  10m   \\
    8  & VHF & 30 a 300 MHz    &  10 $\sim$   1m   \\
    9  & UHF & 300 a 3 000 MHz &   1m $\sim$ 100mm \\
    10 & SHF & 3 a 30 GHz      & 100 $\sim$   10mm \\
    11 & EHF & 30 a 300 GHz    &  10 $\sim$    1mm \\
    12 & THF & 300 a 3 000 GHz &     $<$ 1mm       \\
    \hline
    \end{tabular}
    \label{tablaA}
    \end{center}
    \end{table}

In addition to the above, the bandwidth of the wireless channel is more limited compared to fiber optics \cite{Sharma2014} [6] and conductor cables due to various physical phenomena such as: The presence of cosmic noise \cite{Bala2002,Lanzerotti2005,Nita2002} [7]–[9], atmospheric noise \cite{Reuveni2010,Witvliet2023} [10], [11], atmospheric absorption \cite{Bean1957} [12], multipath interference \cite{Lucas2017} [13], partial reflection of electromagnetic waves in matter, and diffraction \cite{Rossi2000, Blaunstein2003} [14], [15]. These phenomena force communications systems to pre-process signals so that bandwidth is efficient and maintains the possibility of sustaining the desired speeds in an inhospitable scenario so that the receiver acquires a signal-to-noise ratio (SNR) adjusted to current regulations. However, meeting quality standards in a wireless communication system, especially inside any building, is extremely complex in cities because there is a lot of electromagnetic wave shadowing, as there are many buildings of different heights that block others, making it difficult for receivers indoors to have an equitable SNR among different users. This technology addresses challenges in terms of power distribution and SNR, as the wireless signal is affected by different media with which it interacts on its journey from the source of emission to the different signal reception areas. At this point, the conditions for signals in horizontal properties, which are the subject of our study, are directly affected to such an extent that there are places with no coverage due to total signal loss.

\smallskip

Although there are significant and diverse challenges in terms of communication, as mentioned above regarding RF systems, this is not the only path that can be taken in search of a recursive solution. there is also another form of information transmission that can be exploited for indoor environments, such as visible light communication, which is a technology that works very well indoors \cite{Mohsan2023} [16] and has great potential to contribute to solving the problems identified. These systems use light-emitting diodes (LEDs) as transmitters and different types of photodetectors such as solar panels and photodiodes, among others, as receivers. Research on VLC \cite{Almadani2020,Liu2023,Memedi2021,Singh2020} [17]–[20] shows the different applications, and also mentions the advantages of low energy consumption \cite{Khan2017,Matheus2019,Pathak2015} [21]–[23] and the use of free spectrum bands, since these systems operate in the visible range (see Figure 1 \Cref{EspectroVisible}, taken from \cite{DynamoElectronics} [24]). These technical conditions allow us to transmit information at higher speeds and with greater bandwidth  \cite{Kabir2022} [25] compared to radio frequencies. At this point, we observe the need for technological convergence that allows the coexistence of broadcasting systems and this short-range optical communication system, as it has great advantages for implementation in the country.

\smallskip

In Colombia, regulations are enforced through the Technical Regulation on Internal Networks (RITEL) \cite{RITEL} [26], which aims to improve and expand the coverage of telecommunications services in Colombia. It should be noted that this regulation only applies to properties that are subject to a co-ownership or horizontal property regime. Although the main objective of RITEL is to guarantee users' freedom of choice in terms of PRST options and provision of services for digital development. It also sets out the minimum conditions for support infrastructures and the access network for Digital Terrestrial Television (DTT) services. In this context, the development and application of a hybrid system in indoor environments is proposed, as it would exponentially increase the possibility of working hand in hand with IoT devices, and our systems would be able to support new technologies that require higher bandwidths.

\textcolor{blue}{Sumado a lo anterior, el ancho de banda del canal inalámbrico es más limitado, en comparación con la fibra óptica \cite{Sharma2014} y los cables de conductores, debido a diferentes fenómenos físicos como lo son: La presencia de ruido cósmico \cite{Bala2002,Lanzerotti2005,Nita2002}, el ruido atmosférico \cite{Reuveni2010,Witvliet2023}, la absorción atmosférica \cite{Bean1957}, la interferencia por multitrayectoria \cite{Lucas2017}, la reflexión parcial de las ondas electromagnéticas en la materia y la difracción \cite{Rossi2000, Blaunstein2003}. Estos fenómenos fuerzan a los sistemas de comunicaciones a procesar previamente las señales para que el ancho de banda sea eficiente y mantenga la posibilidad de sostener las velocidades deseadas en un escenario inhóspito para que el receptor adquiera una relación señal a ruido (SNR, por sus siglas en inglés) ajustado a la regulación vigente. }
\smallskip
\textcolor{blue}{Sin embargo, lograr cumplir con los estándares de calidad en un sistema de comunicación inalámbrica, sobre todo en los interiores de cualquier edificación, es supremamente complejo en las ciudades debido a que existe mucho ensombrecimiento de las ondas electromagnéticas, pues existen muchos edificios de diferentes alturas que opacan a otros, dificultando así que los receptores en los interiores no tengan una SNR equitativa entre los distintos usuarios. Esta tecnología aborda unos retos en cuanto a distribución de la potencia y la SNR, ya que la señal inalámbrica se ve afectada por distintos medios con los cuales interactua en el trayecto recorrido desde el origen de emisión; hasta las distintas zonas de recepción de la señal. En este punto las condiciones para las señales en propiedades horizontales; las cuales son nuestro objeto de estudio, se ven directamente afectadas a tal punto de que existen lugares de no cobertura por pérdida total de la señal. }

\textcolor{blue}{Aunque hay grandes y diversos retos a nivel de comunicación como se viene mencionando acerca de los sistemas RF, no es el único camino que se puede tomar en busca de una solución recursiva; también existe otra forma de transmisión de información de la cual es posible sacar provecho para entornos dentro de edificaciones, como lo es el caso de la comunicación por luz visible, la cual es una tecnología que funciona muy bien a nivel Indoor \cite{Mohsan2023}, con gran potencial para aportar a dichas problematicas detectadas. Estos sistemas usan como transmisores Diodos emisores de luz (LED, por sus siglas en inglés) y como receptores diferentes tipos de fotodetectores como paneles solares, fotodiodos, entre otros. Las investigaciones sobre VLC \cite{Almadani2020,Liu2023,Memedi2021,Singh2020} muestran las diferentes aplicaciones, tambien se mencionan las ventajas de bajo consumo energético \cite{Khan2017,Matheus2019,Pathak2015}, uso de bandas del espectro libre; ya que estos sistemas operan en el rango visible (Ver Figura \ref{EspectroVisible}, tomada de \cite{DynamoElectronics}). Estas condiciones técnicas permiten que podamos transmitir información con mayor velocidad y con un ancho de banda superior \cite{Kabir2022} comparado con las Radiofrecuencias. En este punto observamos la necesidad de una convergencia tecnológica que permita la coexistencia entre los sistemas de radiodifusión y este sistema de comunicación óptico de corto alcance; ya que tiene grandes ventajas como para ser implementado en el país. }

%%%%%%%%%%%%%%% Figura chingona %%%%%%%%%%%%%%%
\begin{figure}[ht!]
\centerline{\includegraphics[scale=0.36]{figures/VLC-Spectrum.png}}
\caption{Espectro visible.}
\label{EspectroVisible}
\end{figure} 
%%%%%%%%%%%%%%% Figura chingona %%%%%%%%%%%%%%% 

En Colombia es aplicada una regulación a través del Reglamento Técnico de Redes Internas – RITEL \cite{RITEL}, el cual tiene como propósito mejorar y masificar la cobertura de servicios de telecomunicaciones en Colombia. Vale aclarar que esta normativa solo aplica para inmuebles que tengan un régimen de copropiedad o propiedades horizontales. Aunque el objetivo principal de RITEL es garantizar la libre elección en cuanto a las opciones de los PRSTs de parte de los usuarios y la prestación de los servicios para el desarrollo digital. Tambien se dictan las condiciones mínimas para las infraestructuras de soporte y la red de acceso para el servicio de Televisión Digital Terrestre - TDT. En este contexto se propone el desarrollo y la aplicación de un sistema híbrido en ambientes Indoor, ya que aumentaría exponencialmente la posibilidad de trabajar de la mano con dispositivos IoT, y nuestros sistemas estarían en la capacidad de soportar nuevas tecnologías que requieran anchos de banda con mayores necesidades.




\section{STATE OF THE ART} \label{chap:ch2}
% \section{Estado del Arte} \label{chap:ch2}

The demand for wireless communications in indoor environments has increased significantly due to the growth of connected devices and the need for high data transfer rates \cite{Mitra2019,Chowdhury2019} [27], [28]. However, broadcasting and mobile communication services face multiple limitations in these scenarios, such as interference, signal attenuation, and radio spectrum saturation, resulting in poor performance of existing systems \cite{Lee2007,Tan2016} [29], [30]. \Cref{RF} shows a graphical representation of a real-world scenario of multiple signal broadcasting in an urban environment.

% \smallskip 

%%%%%%%%%%%%%%% Figura chingona %%%%%%%%%%%%%%%
\begin{figure*}[ht!]
% \begin{figure}[htbp]
\centerline{\includegraphics[scale=1.5]{figures/figure2.jpg}}
\caption{Radio Frequency Signals.}
\label{RF}
\end{figure*} 
%%%%%%%%%%%%%%% Figura chingona %%%%%%%%%%%%%%% 

These problems are evident in emerging technologies such as 5G and future 6G, and even in digital terrestrial television (DTT), the internet, and satellite television, or any communication service that involves a link from the outside to the inside, where the high frequencies used are susceptible to loss and require a line of sight free of obstacles \cite{Chaves2016,Ribadeneira-Ramírez,Guidotti2019} [31]–[33].

The migration from analog to digital systems and the use of white space spectrum have attempted to mitigate some of these problems, but significant challenges remain in spectral efficiency and spectrum management \cite{Wu2017,Zhang2017} [34], [35]. Given these limitations, it is necessary to explore alternatives that complement traditional radio frequency (RF) technologies and ensure reception and transmission quality inside any building, that is, maintaining the same level of communication quality whether in a home or in more inhospitable settings such as parking lots, shielded environments, tunnels, or underground mines.

VLC is a wireless communication technology that has been dominating scenarios such as indoor communication and could support RF systems \cite{Xu2016,Hao2013} [36], [37]. VLC uses the visible light band of the electromagnetic spectrum to transmit information, taking advantage of existing LED lighting infrastructure and offering benefits such as increased bandwidth, immunity to other radio signals, electromagnetic interference, and improved security \cite{Karunatilaka2015,RAHAIM2015T} [38], [39].

\smallskip 

However, VLC also has its own limitations, such as dependence on line of sight, interference with other light sources, and limited coverage \cite{Rahaim2015,Hosney2020} [40], [41]. Even so, VLC could be a technology that would allow the distribution of outdoor radio signals indoors, ensuring a more uniform signal-to-noise ratio in every corner of any building, especially in urban environments. To achieve this goal, an in-depth analysis of the architecture of RF and VLC communication systems is required to enable hybridization without interfering with the standards that currently govern these technologies.

This chapter analyzes the current limitations of RF and VLC technologies in indoor environments, with the aim of identifying opportunities to improve wireless communication in these spaces. \Cref{seccionRFLimitaciones} addresses the tech-RF technologies and their limitations, covering issues such as spectrum saturation, interference, and specific challenges in the implementation of 5G and 6G networks. \Cref{seccionVLC} focuses on VLC systems, addressing weaknesses such as their dependence on line of sight, optical interference, and challenges in modulation and bidirectional communication. \Cref{seccionConvergencia} proposes an architecture that links the radio signal capture system, the indoor distribution network, and VLC systems, and discusses the conditions and enabling technologies for hybrid RF-VLC systems and their possible application scenarios.

% \subsection{Tecnología RF y sus Limitaciones en Entornos Indoor}
\subsection{RF Technology and Its Limitations in Indoor Environments}
\label{seccionRFLimitaciones}

    RF technologies in indoor environments can experience significant attenuation due to absorption by walls and other obstacles, which is especially problematic at higher frequencies such as the millimeter waves used in 5G \cite{Singh2017} [42]. Attenuation increases in densely populated areas, where multiple reflections and scattering further complicate signal propagation \cite{Echarri} [43]. In these environments where there is human movement, signal blocking by bodies and scattering by objects affect the quality of signal propagation in 28 GHz links \cite{Dalveren2019,Benzaghta2021} [44], [45]. In addition, the coexistence of multiple wireless devices and systems generates electromagnetic interference that can affect network performance, increasing latency and the likelihood of data loss \cite{Chiaramello2019} [46].

    On the other hand, saturation in Internet of Things (IoT) devices will continue to grow. As applications in environments such as smart cities, digital health, and industry continue to expand, IoT systems are forced to handle large volumes of data, facing various limitations, such as energy consumption and data transmission reliability  \cite{Ebenezer2023} [47]. These factors have led to significant saturation, to the point of a shortage of available spectrum, especially in unlicensed bands such as ISM 2.4GHz \cite{Hou2022} [48]. To address these challenges related to signal attenuation and spectrum saturation, it is essential to understand how the radio spectrum is distributed and used in different frequency bands. Below (see \Cref{tabla1}) are some of the technologies that operate in each of these bands. This information allows us to visualize the areas of greatest congestion and the opportunities for implementing new developments that optimize the performance and management of wireless networks.

    
    \begin{table}[ht!]
    \caption{Tecnologías RF en Diferentes Bandas del Espectro}
    \begin{center}
    \begin{tabular}{cccc}
    \hline
    \textbf{\textit{Tecnología}} & \textbf{\textit{Frec (GHz)}}& \textbf{\textit{Saturación}}& \textbf{\textit{Ref}} \\
    \hline
    4G LTE, 5G   & 0.8 - 2.6  & Alta & \cite{Kim2019,Lee2022,Qian2020} \\ 
    WiFi (802.11b/g/n)  &  2.4  & Muy alta  &  \cite{Nasser2021,Ahmad2020}   \\
    5G (banda C) &  3.5 & En aumento & \cite{Lee2018,Hu2018,Frieden2020}  \\
    WiFi (802.11a/ac/ax)   &  5 & Alta & \cite{Abid2023,Avallone2021,Chen2020}  \\
    5G (mmWave) & 24-28 & Baja & \cite{Singh2019,Sung2020,Erofeev2020} \\
    WiGig & 60 & Baja & \cite{Fierro2020,Liu2022,Kim2021} \\
    Punto a punto & 70-80 & Muy Baja & \cite{Larsson2020,Liu2021} \\
    \hline
    \end{tabular}
    \label{tabla1}
    \end{center}
    \end{table}


    To overcome some of the limitations of conventional wireless technologies, it is necessary to explore alternatives in different regions of the electromagnetic spectrum in order to reduce interference and increase efficiency in indoor environments. In the following section, we will discuss visible light communication in detail, analyzing its benefits, limitations, and potential integration with other technologies. to optimize performance in indoor environments.

%%%%%%%%%%%%%%%%%%%%%%%%%%%COPIA TABLA%%%%%%%%%%%%%%%%%%%%%%%%%%%%%%%%%%%%%%%%%%%%%%%%%%%%%%
    \begin{table*}[ht!]
    \caption{Algunos dispositivos VLC y sus posibles aplicaciones.}
    \centering
    % \renewcommand{\arraystretch}{1.3}
    % \begin{tabular}{|c|c|c|>{\raggedright\arraybackslash}p{9cm}|}
    \begin{tabularx}{\textwidth}{cXXc}
    \hline
    \textbf{\textit{Tipo de Tx y/o Rx }} & \textbf{\textit{Descripción}} & \textbf{\textit{Aplicación }} & \textbf{\textit{Ref}} 
    \\ \hline
    RGB LED PAM-4  & Emplea LEDs tricolores para transmitir datos mediante modulación PAM-4. & Sistemas de alta velocidad con baja interferencia. & \cite{Wang2020,Zhou2019}
    \\ \hline
    LED-to-LED VLC  & Se utiliza un LED tanto transmisor como receptor. Esto permite una comunicación de bajo costo. & Aplicaciones de corto alcance y  bajo costo. & \cite{Mir2021,Pramudya2023}  
    \\ \hline
    \makecell{Silicon Photomultiplier \\(SiPM) Receiver}  & Receptor con SiPM que opera con longitudes de onda de 405 nm. Con alta tasa de transmisión de datos y puede soportar hasta 1 Gbps. & Aplicaciones Indoor, sistemas de comunicación rápidos. & \cite{Ali2020,Ma2023} 
    \\ \hline
    \makecell{Fotodetector con\\ lentes Fresnel} & Para mejorar la ganancia óptica. Ofrece baja latencia y transmisión robusta. & Sistemas de transporte inteligente y comunicación vehículo-infraestructura. & \cite{Nawaz2020,Younus2023} 
    \\ \hline
    \makecell{Transmisores LED \\con MOSFET} & Emplea LED y MOSFET como amplificador sumador. Se usa para transmisión en distancias variables con codificación Manchester.& Comunicación vehículo a vehículo e infraestructura. & \cite{Huang2021,Greives2020} 
    \\ \hline
    \makecell{Perovskite Dual-Band \\Photodetector} & Utiliza fotodetectores de perovskita con capacidad de respuesta dual. Puede captar señales de diferentes longitudes de onda. & Comunicación eficiente y de alta velocidad Indoor & \cite{Huang2020} 
    \\ \hline
    \end{tabularx}
    \label{tabla2}
    \end{table*}
    %%%%%%%%%%%%%%%%%%%%%%%%%%%%%%%%%%%%%%%%%%%%%%%%%%%%%%%%%%%%%%%%%%%%%%%%%%%%%%%%%%%%%%

% \subsection{Sistemas de Comunicación por Luz Visible }
    \subsection{Visible Light Communication Systems}
    \label{seccionVLC}

    %%%%%%%%%%%%%%% Figura chingona %%%%%%%%%%%%%%% 
    \begin{figure*}[ht!]
    \centerline{\includegraphics[scale=0.2]{figures/figure3.png}}
    % \caption{Esquema general de un sistema de comunicación por luz visible.}
    \caption{General diagram of a visible light communication system.}
    \label{Esquema General VLC}
    \end{figure*}
    %%%%%%%%%%%%%%% Figura chingona %%%%%%%%%%%%%%% 

    Unlike RF signals, which operate in a regulated and licensed spectrum range, mainly between 3 kHz and 300 GHz, VLC technology operates in the visible range of 400 to 800 THz, a part of the spectrum considerably higher than that allocated to radio communications; this difference in frequency range allows both technologies to coexist without interference. Visible light communication, in addition to being license-free \cite{Mushfique2018} [79], offers benefits such as greater energy efficiency and a greater capacity to transmit data at high speeds \cite{Chi2020} [80]. Compared to RF, VLC uses energy more efficiently, as it can take advantage of existing LED lighting infrastructure, allowing data transmission with almost no additional energy consumption while simultaneously providing illumination \cite{Chamani2022,Nawawi2019} [81], [82]. The transmitters used in VLC systems are LEDs, which can be modulated at high speeds for data transmission \cite{Farid2022,Kong2018} [83], [84], while the receivers use photodetectors such as photodiodes \cite{Bishnu2018} [85] or silicon photomultipliers (SiPM) \cite{Ahmed2019}  [86], designed to capture modulated light and decode information. \Cref{tabla2} presents some of these devices for visible light communication along with their possible technical applications.

    % \subsection{ Tecnología LED }
    \paragraph{LED technology}
    Light-emitting diodes have transformed modern lighting thanks to their energy efficiency and durability. Physically, they are PN-type junction semiconductor devices, where the P junction is doped with acceptor atoms and the N junction is doped with donor atoms, which, when directly polarized, allow current to flow in one direction through the semiconductor material, causing radioactive recombination of electrons and holes in the depletion region; This causes photons to be emitted in the form of light \cite{Cengiz2022} [87]. LEDs can be classified according to the semiconductor material used \cite{Rahman2014} \cite{Marciniak2019} [88] [89] or their specific application. Among the most common types are infrared LEDs, used mainly in remote controls and proximity sensors. On the other hand, red, green, and blue (RGB) LEDs have been fundamental in the development of displays \cite{Wang2022} [90] and adjustable lighting systems, where the mixture of these colors allows for a wide range of colors \cite{Muthu2002} [91]. The emission color of LEDs is related to semiconductor materials with which they are manufactured; \Cref{LED Semiconductor Materials} provides a brief overview of this.
    

    \begin{table}[ht!]
        \centering
        \caption{Materiales Semiconductores de los LEDs, tomado de \cite{Ghassemlooy2017}.}
        \begin{tabular}{lcl}
        \toprule
        \textbf{Color}       & \textbf{$\lambda$ (nm)} & \textbf{Material Semiconductor}                                \\
        \midrule
        \addlinespace[3pt]
        
        Infrarrojo             & $\lambda > 760$           & GaAs, AlGaAs                                                     \\
        Rojo                  & $610 < \lambda < 760$     & AlGaAs, GaAsP, AlGaInP, GaP                                       \\
        Naranja               & $590 < \lambda < 610$     & GaAsP, AlGaInP, GaP                                               \\
        Amarillo               & $570 < \lambda < 590$     & GaAsP, AlGaInP, GaP                                               \\
        Verde                & $500 < \lambda < 570$     & InGaN/GaN, GaP, AlGaInP, AlGaP                                    \\
        Azul                 & $450 < \lambda < 500$     & ZnSe, InGaN                                                       \\
        Violeta               & $400 < \lambda < 450$     & InGaN                                                             \\
        Ultravioleta          & $\lambda < 400$           & Diamante, AlGaN, AlGaInN                                           \\
        \bottomrule
        \end{tabular}
        \label{LED Semiconductor Materials}
    \end{table}

    %%%%%%%%%%%%%%%%%%%%%

    White LEDs, essential in general lighting, can be obtained using two main methods: combining RGB LEDs or using blue LEDs coated with phosphor. In the first case, white light is generated by mixing three individual primary color LEDs (RGB), allowing f o r precise adjustments in color reproduction. In the second method, a blue LED excites a layer of yellow phosphor, generating highly efficient, low-cost white light, which has become the dominant technology in the lighting market. 
    
    Other relevant types include ultraviolet LEDs, widely used in sterilization processes, substance detection, and material curing in industrial and medical applications. Finally, high-brightness and high-power LEDs have been designed for demanding applications, such as architectural lighting and the automotive industry, where high light intensity and high resistance to adverse environmental conditions are required. The response of LEDs is presented below in general terms, given that it is key to consider that in any VLC system, the transmission process will depend directly on this characteristic.

    % \subsection{ Respuesta en Frecuencia del LED}
    \paragraph{LED Frequency Response}
    White light can be reproduced using LED technology, either through a combination of RGB LED luminaires or blue LEDs with one or more layers of phosphor. Commercially, the most common are blue LED bulbs with a layer of yellow phosphor \cite{Li2022} [93], which is used to convert the blue light emitted by the LED chip into white light. This is achieved by emitting yellow light which, when combined with blue light, produces high-quality white light. These LEDs have the advantage of being cheaper, easier to manufacture, and having good luminous efficiency, offering a simple and efficient way to generate white light, albeit with a slight reduction in the color rendering index \cite{Chen2019} [94]. 

    A frequency response of around 2 MHz is commonly observed in white LEDs based on blue LED technology with yellow phosphor, as reported in various studies \cite{Chen2019} [94]. This behavior is mainly due to the decay time of the yellow phosphor, which is on the order of 80 ns, resulting in a 3 dB bandwidth close to 2 MHz. This parameter is crucial in determining the modulation capability of these devices for indoor visible communications applications \cite{Wang2024,Park2004,Sun2015} [95]–[97]. Furthermore, the modulation offered by these LEDs is perfectly aligned with the speed and light quality requirements necessary for commercial visible communications systems, while simultaneously enabling high light efficiency \cite{Hu2015} [98] and a good color rendering index \cite{Zhang2018} [99].

    %%%%%%%%%%%%%%% Figura chingona %%%%%%%%%%%%%%% 

    \begin{figure*}[ht!]
    \centerline{\includegraphics[scale=0.5]{figures/figure4.jpg}}
    \caption{LOS, NLOS, Diffuse Link, Quasi-diffuse. Estudio tomado de  \cite{Rahman2018}}
    \label{LOS_NLOS}
    \end{figure*}
    
    %%%%%%%%%%%%%%% Figura chingona %%%%%%%%%%%%%%% 
    
    
    % \subsection{Dependencia de LOS y Desafíos de Interferencia en Sistemas VLC}
    \paragraph{Dependence on LOS and Interference Challenges in VLC Systems}
    Visible light communication systems rely heavily on Line of Sight (LOS) to achieve optimal performance between the transmitting LED and the receiver. This is key, as diffuse or non-directional configurations limit the achievable rate. LOS increases the intensity of the received signal and implies a higher SNR, which allows for an increase in the data rate or link distance and reduces the risk of Inter-Symbol Interference (ISI) \cite{Ghassemlooy2017} [92]. When obstructions are present, communication links can suffer a significant drop in performance. This LOS dependence has motivated studies to improve blocking tolerance, either through reflective surfaces or beam angle optimization technologies \cite{Cao2019,Hamad2023} [101], [102].
    
    % %%%%%%%%%%%%%%% Figura chingona %%%%%%%%%%%%%%% 
    In \Cref{LOS_NLOS} Figure 4, the authors mention four possible configurations (LOS, NLOS, Diffuse, Quasi-Diffuse) which are key to channel modeling and a correct description of a VLC system \cite{Rahman2018} [100]. \Cref{tabla3} summarizes the main issues previously mentioned in visible light communication systems. To mitigate the negative effects of channel distortion and inter-symbol interference in communication systems, various compensation techniques have been developed. These techniques seek to maintain link quality and stability, improving system performance. Next, equalization will be addressed as one of the techniques relevant to our research.


    \begin{table}[ht!]
    \caption{Challenges in VLC Links}
    \begin{center}
    \begin{tabular}{ccc}
    \hline
    \textbf{\textit{Desafío}} & \textbf{\textit{Descripción}}& \textbf{\textit{Ref}} \\ 
    \hline
    \makecell{Dependencia\\ de LoS}    & \makecell{VLC requiere una trayectoria \\sin obstrucciones entre el\\ transmisor y el receptor, \\lo que limita su uso en \\ambientes con obstáculos. } & \cite{Rodoplu2020,Zhou2023} \\ 
    \hline
    \makecell{Interferencia \\Óptica}  & \makecell{La luz ambiental y fuentes de luz\\ artificial causan interferencia \\en los enlaces VLC, afectando \\la claridad de la señal. } & \cite{Forkel2019,affan2020}   \\
    \hline
    \makecell{Desafíos en \\Comunicación \\Bidireccional}  & \makecell{El control bidireccional en VLC es\\ complejo debido a la alineación \\estricta de los dispositivos transmisores\\ y receptores para mantener la LOS. } &\cite{Almohanna2019,Wei2018}  \\
    \hline
    \end{tabular}
    \label{tabla3}
    \end{center}
    \end{table}
    
% \subsection{Técnicas de Compensación de Errores}
\paragraph{Error Compensation Techniques}
Error compensation techniques are fundamental in communications systems, as they allow for the mitigation of adverse effects of the transmission channel, such as noise, dispersion, and interference \cite{Djordjevic2022} [109], thus ensuring integrity of the information received. Among the various existing techniques, the following are particularly noteworthy: channel error correction (Error Correction Coding, ECC) \cite{Mambou2021} [110] and equalization \cite{Metzger2011} [111]. Where ECC adds redundancy to the transmitted data, allowing errors to be detected and corrected in the receiver, while equalization combats distortion by adjusting the signal to counteract unwanted effects \cite{Nguyen2019} [112].

In optical communications, and especially in VLC, error compensation is particularly important due to the variable and unpredictable nature of the optical channel \cite{Mambou2021} [110], which is affected by multipath dispersion, noise, and the limitations of the transmitting and receiving devices \cite{Shapiro2023} [113]. For this reason, pre-equalization and post-equalization techniques are used, and increasingly, hybrid methods that combine both strategies \cite{Cenqin2021} [114].

% \subsection{Ecualización}
\paragraph{Equalization}

Equalization is an essential technique in communication systems, whose objective is to counteract the distortions and limitations imposed by the transmission channel. These distortions may be due to the limited frequency response of the components, temporal dispersion, the non-linearity of the devices (such as LEDs in VLC), and interference between symbols. Equalization improves the quality of the received signal, increases the data transmission rate, and reduce the bit error rate (BER). There are different equalization strategies: pre-equalization, post-equalization, and combined schemes, each with specific advantages and limitations. The choice of the appropriate technique depends on the characteristics of the channel, the complexity allowed in the transmitter and receiver, and the performance requirements of the system.

The combination of pre- and post-equalization can offer additional improvements in system robustness and efficiency, especially in channels with high dispersion or nonlinearity \cite{Siuzdak2022,Singh2022} [115], [116].

% %%%%%%%%%%%%%%% Figura chingona %%%%%%%%%%%%%%% 

    \begin{figure*}[ht!]
    \centerline{\includegraphics[scale=0.34]{figures/RF-VLC/figure5.pdf}}
    % \caption{Amplificador RF y Filtro pasabanda en la entrada del receptor. Tomado de \cite{Carr2000}}
    \caption{Receptor Heterodineo. Tomado de \cite{Carr2000}.}
    \label{Heterorinaje-RF}
    \end{figure*}
    
%     %%%%%%%%%%%%%%% Figura chingona %%%%%%%%%%%%%%% 

% \subsubsection{Pre-ecualización \& Pos-ecualización }
% \paragraph{Pre-ecualización \& Pos-ecualización }
\paragraph{Pre-equalization \& Post-equalization}

Pre-equalization is implemented in the transmitter and consists of modifying the signal before sending it, anticipating and compensating for foreseeable channel distortions. In optical and VLC systems, it is especially useful for counteracting the limited bandwidth and non-linearity of LEDs, thus allowing higher transmission rates and reducing complexity in the receiver \cite{Surampudi2022} [118]. Digital pre-equalization schemes have been shown to increase effective bandwidth and improve noise tolerance, achieving significant increases in data rate without the need for post-equalization \cite{Li2022} [93]. However, their effectiveness depends on precise knowledge of the channel and may be limited by the capacity of the pre-equalization circuits, especially in analog implementations \cite{Yan2020} [119].

On the other hand, post-equalization is applied in the receiver and uses adaptive filters or advanced algorithms to correct distortions affecting the received signal, such as ISI caused by temporal or modal dispersion. It is particularly useful when the channel is variable or difficult to model in the transmitter, although it can amplify noise and increase the complexity of the receiver \cite{Ge2020} [120]. In highly nonlinear channels or channels with a high signal-to-noise ratio, post-equalization can outperform pre-equalization in terms of received signal quality. Furthermore, the combination of both techniques (pre- and post-equalization) has been shown to improve the overall performance of the system, reducing the power required and improving the sensitivity of the receiver \cite{Siuzdak2019} [121].

% \subsubsection{Técnicas híbridas y tendencias recientes}
\paragraph{Hybrid techniques and recent trends}
Currently, VLC research is exploring hybrid approaches that combine pre-equalization and post-equalization, as well as advanced machine learning-based algorithms, to achieve greater robustness and flexibility \cite{LiangQiao2018,Chi2018} [122], [123]. For example, neural networks and deep learning are being applied to design adaptive equalizers that can significantly improve performance in changing environments \cite{Guan2018,Lu2025,Li2023} [124]–[126]. Furthermore, the trend toward integrating VLC with other technologies such as RF (e.g., hybrid RF-VLC systems) makes the study of adaptive and intelligent error compensation techniques even more relevant \cite{Sun2023,daSilva2024,Abdallah2025} [127]–[129]. In other words, the efficient treatment of errors in VLC systems depends on the appropriate combination of pre-equalization, post-equalization, and hybrid techniques, adjusted to the environment and the nature of the optical channel, which is crucial for the viability of robust RF-VLC hybrid systems.
    
% \newpage
\subsection{Convergence between RF \& VLC Systems}
    \label{seccionConvergencia}

 The convergence between radio frequency and visible light communication technologies is presented here not as a generic solution, but as a specific response to the need to capture an RF signal in an open environment and retransmit it efficiently in a confined space through an optical optical system. This proposal, which combines system Reception by antenna with retransmission using modulated light is based on principles widely described in the literature on superheterodyne receivers and the generalities of indoor VLC systems (see Figure 5 and Figure 6, respectively), where the use of mixers and local oscillators allows the signal frequency to be adapted for subsequent processing without loss of fidelity \cite{Carr2000} [117]. By integrating these elements with a VLC optical channel, the aim is to Take advantage of low interference, high security, and the growing availability of indoor LED infrastructure to operate in indoor spaces \cite{Karunatilaka2015} [38].

Additionally, to minimize errors and facilitate component integration, the use of pre-assembled modules is proposed, such as mixers, low-noise amplifiers (LNA), and Bias-T coupling circuits, which ensure the operational stability of the system during testing. This research is based on the premise that it is possible not only to combine the advantages of both technologies, but also to design a continuous and functional signal flow that maintains a uniform SNR even in physically hostile environments. The technical description of the heterodyning process and the detailed implementation of the hybrid system are developed in the following subsection.

    %%%%%%%%%%%%%%% Figura chingona %%%%%%%%%%%%%%% 

    \begin{figure*}[ht!]
    \centerline{\includegraphics[scale=0.35]{figures/RF-VLC/figure6.pdf}}
    \caption{Generalidad de los Sistemas VLC-Indoor \cite{Karunatilaka2015}.}
    \label{VLC-Indoor}
    \vspace{10pt} % Espacio opcional entre la leyenda y la línea
    \rule{\textwidth}{0.2pt} % Mismo ancho que la imagen
    \vspace{8pt} % Espacio opcional entre la leyenda y la línea
    \end{figure*}
    
    %%%%%%%%%%%%%%% Figura chingona %%%%%%%%%%%%%%% 
    
% \subsection{Sistema Híbrido Propuesto}
\paragraph{Proposed Hybrid System}
This paper proposes a hybrid communications system that integrates RF technology with a VLC system. The approach is based on a review of the specialized literature and the various studies found, which highlight the complementarity between both domains for scenarios with high spectral demand and localized security \cite{BravoAlvarez2023} [130]. Inspired by these approaches, a design is proposed that takes advantage of the penetration and robustness of RF, together with the directionality, physical security, and wide free spectrum offered by VLC \cite{Wang2023} [131]. The system consists of two main stages: the first stage is the capture and processing of RF signals for conversion to the optical domain; and the second stage is responsible for receiving and demodulating the VLC signal to recover the base information. The range specifications and characteristics of the equipment are presented in \Cref{Componentes del Transmisor Híbrido.} \& \Cref{Componentes del Receptor Híbrido}.
% \smallskip
\paragraph{Transmission Stage of the RF–VLC Hybrid System}
In this stage, the RF signal is picked up by an antenna dipole coupled to a USRP 2900; which provides the necessary flexibility for signal processing. This signal is then connected to a low-noise amplification block; it is then fed to an AD831 mixer, which, in conjunction with a local oscillator,
implements RF heterodyning, shifting the signal to a desired intermediate frequency (IF). It is then coupled to the optical emitter via a bias tee, which is used to apply a DC signal necessary to polarize and turn on the LED, which is used as an optical modulator in an IM/DD scheme to emit the signal in
the visible light domain \cite{Yeh2015} [132]. 

    %%%%%%%%%%%%%%% Figura chingona %%%%%%%%%%%%%%% 
    % \begin{figure*}[ht!]
    \begin{figure*}[ht!]
    \centerline{\includegraphics[scale=0.565]{figures/RF-VLC/figure7.jpg}}
    \caption{Transmisor Híbrido Propuesto.}
    \label{Tx-Hibrido}
    % \vspace{10pt} % Espacio opcional entre la leyenda y la línea
    \rule{\textwidth}{0.2pt} % Mismo ancho que la imagen
    \vspace{20pt} % Espacio opcional entre la leyenda y la línea
    \end{figure*}
    %%%%%%%%%%%%%%% Figura chingona %%%%%%%%%%%%%%% 
    %%%%%%%%%%%%%%% Figura chingona %%%%%%%%%%%%%%% 
    \begin{figure*}[ht!]
        \centerline{\includegraphics[scale=0.5]{figures/RF-VLC/figure8.jpg}}
        \caption{Rx Híbrido.}
        \label{Rx-Hibrido}
        % \vspace{10pt} % Espacio opcional entre la leyenda y la línea
        % \rule{\textwidth}{0.2pt} % Mismo ancho que la imagen
        % \vspace{22pt} % Espacio opcional entre la leyenda y la línea
    \end{figure*}
%%%%%%%%%%%%%%% Figura chingona %%%%%%%%%%%%%%% 
% \smallskip
% \subsubsection{Etapa de Recepción del Sistema Híbrido RF-VLC} 
\paragraph{RF-VLC Hybrid System Reception Stage} 
The optical signal emitted by the LED is captured by a ThorLabs PDA25K(-EC) sensor, chosen for its high sensitivity in the visible range. In this reverse process, it is necessary to extract the DC component of the signal, which is done using a Bias Tee (with the same technical specifications as the transmission stage). The signal is then amplified with an LNA module identical to the one used in the transmission stage to maintain symmetry in the processing chain. 
Next, the previous block is connected to an AD831 mixer which, together with a USRP 2901 as a local oscillator, performs the inverse frequency conversion, allowing the original base signal to be recovered. Finally, the recovered signal is connected to an RTL2832U, a device widely used for reception testing in various RF bands. This design is based on the use of pre-assembled electronic components, whose benefits include accelerated construction and experimental validation, high compatibility with open-source software platforms, and a reliable basis for academic experimentation.
The proposed hybrid system offers a functional and reproducible solution that articulates the conversion of RF signals to the VLC domain and their subsequent recovery, opening the door to new experimental applications. The technical details can be seen in Table VII. The following section describes the entire methodological process of technological integration. 

% %%%%%%%%%%%%%%% Figura chingona %%%%%%%%%%%%%%% 
%     \begin{figure*}[ht!]
%         \centerline{\includegraphics[scale=0.6]{figures/RF-VLC/figure8.jpg}}
%         \caption{Rx Híbrido.}
%         \label{Rx-Hibrido}
%     \end{figure*}
% %%%%%%%%%%%%%%% Figura chingona %%%%%%%%%%%%%%% 

%%% OTRA VERSION DE TABLAS
\begin{table*}[ht!]
    \centering
    \footnotesize
    \renewcommand{\arraystretch}{1.3}

    %------------ TABLA IZQUIERDA (TX) -----------------
    \begin{minipage}[t]{0.48\textwidth}
        \caption{\textbf{Componentes del Transmisor Híbrido.}}
        \label{tab:tx_hibrido}
        \centering
        \begin{tabular}{@{}p{3cm}>{\raggedright\arraybackslash}p{5cm}@{}}
        \hline
        \textbf{\textit{Dispositivo}} & \textbf{\textit{Descripción}} \\ 
        \hline
        Low-Noise Amplifier  & Amplificador RF banda ancha de alta ganancia (30\,dB), 
        con rango de operación de 0,1--2000\,MHz. \\
        \hline
        Mixer - AD831 & Mezclador de un solo chip de amplio rango dinámico y baja distorsión, 
        opera entre 0,1--500\,MHz con 10\,dBm de ganancia, con condensadores de aislamiento DC. \\
        \hline
        Bias Tee & RF Microondas HF Bias Tee DC Bias 50K--60\,MHz. Fuente de alimentación de antena activa. \\
        \hline
        LED Azul con Fósforo Amarillo & Linterna de luz blanca convencional con una respuesta en frecuencia típica de 2\,MHz. \\
        \hline
        USRP 2900, 2901 & Del fabricante National Instruments con un rango de operación RF 
        entre los 70\,MHz--6\,GHz. \\
        \hline
        \end{tabular}
    \end{minipage}
    \hfill
    %------------ TABLA DERECHA (RX) -----------------
    \begin{minipage}[t]{0.48\textwidth}
        \caption{\textbf{Componentes del Receptor Híbrido.}}
        \label{tab:rx_hibrido}
        \centering
        \begin{tabular}{@{}p{3cm}>{\raggedright\arraybackslash}p{5.7cm}@{}}
        \hline
        \textbf{\textit{Dispositivo}} & \textbf{\textit{Descripción}} \\ 
        \hline
        Sensor Óptico ThorLabs & Del fabricante ThorLabs con referencia PDA25K(-EC) y ganancia ajustable.\\
        \hline
        Bias Tee & RF Microondas HF Bias Tee DC Bias 50K--60\,MHz. Fuente de alimentación de antena activa.\\
        \hline
        Low-Noise Amplifier & Amplificador RF banda ancha de alta ganancia (30\,dB), 
        con rango de operación de 0,1--2000\,MHz.\\
        \hline
        Mixer AD831 & Mezclador de un solo chip de amplio rango dinámico y baja distorsión, 
        opera entre 0,1--500\,MHz con 10\,dBm de ganancia, con condensadores de aislamiento DC.\\
        \hline
        RTL2832U & El RTL2832U es un chip popular utilizado en dongles de software-defined radio (SDR).\\
        \hline
        USRP 2900 & Del fabricante National Instruments con un rango de operación RF 
        entre los 70\,MHz--6\,GHz.\\
        \hline
        \end{tabular}
    \end{minipage}
\end{table*}

%%%%%%%%%%%%%%%%%%%%%%%%%%%%%%%%%%%%%%%%%%%%%%

% \begin{table*}[ht!]
%     \caption{\textbf{Componentes del Transmisor Híbrido.}}
%     \centering
%     \footnotesize
%     \renewcommand{\arraystretch}{1.3}
%     \begin{tabular}{@{}p{3cm}>{\raggedright\arraybackslash}p{5cm}@{}}
%     \hline
%     \textbf{\textit{Dispositivo}} & \textbf{\textit{Descripción}} \\ 
%     \hline
%     Low-Noise Amplifier  & Amplificador RF banda ancha de alta ganancia (30\,dB), con rango de Operación de 0,1--2000 MHz. \\
%     \hline
%     Mixer - AD831 & Mezclador de un solo chip de amplio rango dinámico y baja distorsión, opera entre 0,1--500\,MHz con 10\,dBm de ganancia, con condensadores de aislamiento DC. \\
%     \hline
%     Bias Tee & RF Microondas HF Bias Tee DC Bias 50K--60\,MHz. Fuente de alimentación de antena activa. \\
%     \hline
%     LED Azul con Fósforo Amarillo & Linterna de luz blanca convencional con una respuesta en frecuencia típica de 2MHz. \\
%     \hline
%     USRP 2900, 2901 & Del fabricante National Instruments con un rango de operación RF entre los 70\,MHz--6\,GHz. \\
%     \hline
%     \end{tabular}
%     \label{Componentes del Transmisor Híbrido.}
% \end{table*}
% %%%%%%%%%%%%%%%%%%%%%%%%%

% %%%%%%%%%%%%%%%%%%%%%%%%%%%%%%%%%%%%%%%%%%%%%%%%%
% \begin{table*}[ht!]
% % \begin{table}[ht!]
% % \begin{table}[H]
%     \caption{\textbf{Componentes del Receptor Híbrido.}}
%     \centering
%     \footnotesize
%     \renewcommand{\arraystretch}{1.3}
%     \begin{tabular}{@{}p{3cm}>{\raggedright\arraybackslash}p{5cm}@{}}
%     \hline
%     \textbf{\textit{Dispositivo}} & \textbf{\textit{Descripción}} \\ 
%     \hline
%     Sensor Óptico ThorLabs & Del fabricante ThorLabs con referencia PDA25K(-EC) y ganancia ajustable.\\
%     \hline
%     Bias Tee & RF Microondas HF Bias Tee DC Bias 50K--60\,MHz. Fuente de alimentación de antena activa.\\
%     \hline
%     Low-Noise Amplifier & Amplificador RF banda ancha de alta ganancia (30\,dB), con rango de Operación de 0,1--2000 MHz.\\
%     \hline
%     Mixer AD831 & Mezclador de un solo chip de amplio rango dinámico y baja distorsión, opera entre 0,1--500\,MHz con 10\,dBm de ganancia, con condensadores de aislamiento DC.\\
%     \hline
%     RTL2832U & El RTL2832U es un chip popular utilizado en dongles de software-defined radio (SDR)\\
%     \hline
%     USRP 2900 & Del fabricante {National Instruments} con un rango de operación RF entre los 70\,MHz--6\,GHz.\\
%     \hline
%     \end{tabular}
%     \label{Componentes del Receptor Híbrido}
% \end{table*}
% %%%%%%%%%%%%%%%%%%%%%%%%%%%%%%%%%%%%%%%%%%%%%%%%%%%%%%%%%%%%%%%%


\clearpage
\section{METHODOLOGY} \label{chap:ch3}

%%%%%%%%%%%%%%% Figura chingona %%%%%%%%%%%%%%%
\begin{figure*}[ht!]
\centerline{\includegraphics[scale=0.55]{figures/figure9.pdf}}
  \caption{Señal RF tomada desde el exterior y conectada al sistema VLC.}
  \label{Antenayagi}
\end{figure*} 
%%%%%%%%%%%%%%% Figura chingona %%%%%%%%%%%%%%% 

This chapter describes in detail the methodology used to achieve convergence between a radio broadcast link and a visible light communication system. The process begins with the capture of the commercial FM signal located at 103.5 MHz in the city of Medellin through a Yagi antenna installed on the fourth floor terrace of the ITM Fraternidad university campus (see diagram in \Cref{Antenayagi}). This signal was connected directly to a USRP, which, together with GNU Radio, allowed it to be processed and subsequently retransmitted at a frequency of 140 MHz.

\subsection{Integration of Broadcasting and Visible Light Communication}
At each stage of the circuit, measurements were taken using a TinySA portable reference spectrum analyzer to verify the performance of the system. As shown in \Cref{fig:Senal-In-Amplificador-Tx-140MHz}, a signal of -52.5 dBm without a DC component was recorded at the USRP output. After passing through the low-noise amplifier, this signal increased to -11 dBm, which can be clearly seen in \Cref{fig:Senal-Out-Amplificador-Tx-140MHz}. Simultaneously, using a second USRP, a 142 MHz signal was generated as the Local Oscillator frequency, which was combined with the 140 MHz signal using the AD831 mixer, resulting in two frequency components: 2 MHz (difference) and 282 MHz (sum). The 2 MHz component was chosen (see \Cref{fig:Senal-Out-Mixer-Tx-2MHz}) because it was suitable for the typical bandwidth of a conventional LED and simplified optical modulation. This signal, called the intermediate frequency (IF), had a power of -16.1 dBm, indicating an attenuation of approximately 5 dB compared to the previous stage. Finally, the signal was sent to a Bias-T circuit, which provided a DC level to excite the light source without interfering with the information. At the Bias-T output, the measured power was -17.1 dBm (see \Cref{fig:Senal-Out-BiasT-Tx-2MHz}), subsequently connecting to a 3W blue LED with yellow phosphor (obtained from a commercial LED flashlight), thus modulating its light intensity according to the electrical signal received.

% %%%%%%%%%%%%% Imagenes junstas super chingonas I %%%%%%%%%%%%%%%
\begin{figure*}[ht!]
    \centering
    % Primera fila
    \begin{subfigure}{0.45\textwidth}
        \centering
        \includegraphics[scale=0.6]{figures/figure10/figure10a.jpg}
        % \caption{Señal a la salida de la USRP, filtrada y retransmitida a 140 MHz.}
        \caption{Signal at the USRP output, filtered and retransmitted at 140 MHz.}
        \label{fig:Senal-In-Amplificador-Tx-140MHz}
    \end{subfigure}
    \hfill
        \begin{subfigure}{0.45\textwidth}
        \centering
        \includegraphics[scale=0.6]{figures/figure10/figure10b.jpg}
        % \caption{Señal después del amplificador LNA, utilizada para amplificar la señal recibida.}
        \caption{Signal after the LNA amplifier, used to amplify the received signal.}
        %\caption{Señal Out Amplificador Tx 140MHz.}
        \label{fig:Senal-Out-Amplificador-Tx-140MHz}
    \end{subfigure}
    
    % Segunda fila
    \vspace{0.5cm}
    \begin{subfigure}{0.45\textwidth}
        \centering
        \includegraphics[scale=0.6]{figures/figure10/figure10c.jpg}
        % \caption{Señal a la salida del mezclador, la cual corresponde a la señal de entrada al Bias-T.}
        \caption{Signal at the mixer output, which corresponds to the input signal to the Bias-T.}
        %\caption{Señal Out Mixer Tx 2MHz.}
        \label{fig:Senal-Out-Mixer-Tx-2MHz}
    \end{subfigure}
    \hfill
    \begin{subfigure}{0.45\textwidth}
        \centering
        \includegraphics[scale=0.6]{figures/figure10/figure10d.jpg}
        % \caption{Señal a la salida del Bias-T, que es la misma señal de entrada al LED.}
        \caption{Signal at the Bias-T output, which is the same as the LED input signal.}
        \label{fig:Senal-Out-BiasT-Tx-2MHz}
    \end{subfigure}
    \vspace{0.3cm}
    \caption{\textbf{Signals at Different Stages of the Hybrid Transmitter}, clearly showing how the signal power levels vary as they pass through the different stages of the hybrid transmitter. It starts with a power of approximately -52.5 dBm at the USRP output, passing through the amplification and frequency conversion processes, until it reaches a level of -17.1 dBm after the Bias-T, corresponding to the intermediate frequency.}

    \label{fig:ConjuntoSenales}
\end{figure*}

% \clearpage

During the reception stage (see \Cref{Rx-Hibrido}), the received signal was demodulated. For this purpose, a ThorLabs PDA25K(-EC) optical reference sensor with variable gain was used, which, according to the manufacturer's specifications, has an effective area of 4.8 mm² and operates in the range of 150 nm to 550 nm, with best sensitivity around 440 nm, within a temperature range between 0°C and 40°C. With a separation distance of 31 cm between the hybrid transmitter (Tx-Hybrid) and the hybrid receiver (Rx-Hybrid), the signal shown in  \Cref{fig:Senal-Rcibida-Sin-Lente-140MHz} was initially captured. It should be noted that this measurement was taken at the end of the Rx-Hybrid module, recording an approximate frequency of 140 MHz with a power of -53 dBm, as a reference for later evaluating the effect of incorporating a Fresnel lens to improve the SNR ratio. 

To analyze the impact of using the aforementioned Fresnel lens, detailed measurements were taken at each stage of the receiver. This lens It was located 13 cm from the ThorLabs optical sensor. It ThorLabs optical was captured sensor initially the optical signal at 2 MHz, after which the DC component was extracted using a Bias-T. As can be seen in \Cref{fig:Senal-In-Amplificador-Rx-2MHz}, the signal at this stage reached -63.1 dBm. Subsequently, the signal entered the low-noise amplifier, registering an output power of -33.6 dBm (see \Cref{fig:Senal-Out-Amplificador-Rx-2MHz}), consistent with the expected gain of the device (approximately 30 dB); the slight difference of 0.5 dBm could be attributed to losses in connections and cables. Continuing with the proposed scheme, the frequency change was performed using an AD831 mixer combined with a signal generated by a 142 MHz local oscillator, again obtaining a signal around 140 MHz. As shown in \Cref{fig:Senal-Rcibida-Lente-140MHz}, the recorded power was approximately -35 dBm, significantly better than the -53 dBm previously observed without the use of the Fresnel lens. This phase successfully concluded the integration of the proposed hybrid system. Additional details will will be presented in the results chapter. The next phase methodology will focus on the statistical analysis of the system.

%%%%%%%%%%%%%%% Imagenes junstas super chingonas II %%%%%%%%%%%%%%%%%
\begin{figure*}[b!]
    \centering
        \begin{subfigure}{0.45\textwidth}
            \centering
            \includegraphics[scale=0.6]{figures/figure11/figure11a.jpg}
            % \caption{Señal recibida en la etapa final de recepción sin usar lente de Fresnel.}
            \caption{Signal received in the final reception stage without using a Fresnel lens}
            \label{fig:Senal-Rcibida-Sin-Lente-140MHz}
        \end{subfigure}
    \hfill
        \begin{subfigure}{0.45\textwidth}
            \centering
            \includegraphics[scale=0.6]{figures/figure11/figure11b.jpg}
            % \caption{Señal medida a la salida del Bias T, ya sin componente DC usando lente de Fresnel.}
            \caption{Signal measured at the output of Bias T, now without DC component, using a Fresnel lens.}
            \label{fig:Senal-In-Amplificador-Rx-2MHz}
        \end{subfigure}

    %segunda fila
    \vspace{0.3cm}
        \begin{subfigure}{0.45\textwidth}
            \centering
            \includegraphics[scale=0.6]{figures/figure11/figure11c.jpg}
            % \caption{Señal a la salida del amplificador  de bajo ruido, usando lente de Fresnel.}
            \caption{Signal at the output of the low-noise amplifier, using a Fresnel lens.}
            \label{fig:Senal-Out-Amplificador-Rx-2MHz}
        \end{subfigure}
    \hfill
        \begin{subfigure}{0.45\textwidth}
            \centering
            \includegraphics[scale=0.6]{figures/figure11/figure11d.jpg}
            % \caption{Señal a la entrada del Doungle RTL2832U, usando lente de Fresnel.}
            \caption{Signal at the input of the RTL2832U dongle, using a Fresnel lens.}
            \label{fig:Senal-Rcibida-Lente-140MHz}
        \end{subfigure}
        % \vspace{0.5cm}
        \caption{\textbf{Signals in the receiver by stages.} Here, two clearly differentiated cases can be observed, considering a fixed distance between the transmitter and receiver of 134 cm. The first image corresponds to the signal measured at the end of the reception stage without using a Fresnel lens. The following three images show the measurements taken at different internal stages of the proposed hybrid receiver, including again the signal at the end of the receiver circuit, but now using a Fresnel lens placed at a distance of 13 cm from the photodetector. It should be noted that the total distance of the optical link remained constant in both cases.}
    \label{fig:ConjuntoSenales}
\end{figure*}

%%%%%%%%%%%%%%% Figura chingona %%%%%%%%%%%%%%% 
    \begin{figure*}[ht!]
    \centerline{\includegraphics[scale=0.45]{figures/figure12.png}}
    \caption{ITM University Institution, Fraternidad Campus.}
    % \caption{Institución Universitaria ITM, Campus Fraternidad.}
    \label{Campus Fraternidad}
    \end{figure*}
%%%%%%%%%%%%%%% Figura chingona %%%%%%%%%%%%%%% 

\subsection{Experimental Laboratory on SNR Degradation of an Urban Broadcasting Signal}\label{sec:se32}
In order to analyze how the signal-to-noise ratio (SNR) varies under different urban conditions, a study was conducted using the same commercial signal previously characterized in previous experiments, specifically at 103.5 MHz, within the urban context of Medellín. The analysis was carried out using a USRP device complemented by AIRSPY software. Strategic locations on the Fraternidad campus of the ITM University Institution (as shown in \Cref{Campus Fraternidad}) were chosen to take measurements, ranging from the basements (levels -1 and -2) to the terrace on the sixth floor, including the intermediate floors, which allowed for a comparative evaluation of the signal propagation conditions.

\clearpage

\section{Resultados} \label{chap:ch4}

\begin{table*}[ht!]
    \caption{{Datos estadísticos SNR (dB) de los diferentes niveles.}}
    \centering
    \renewcommand{\arraystretch}{1.2}
    \begin{tabular*}{\textwidth}{@{\extracolsep{\fill}}ccccccccccc@{}}
    \hline
    \textbf{\textit{Lugar}} & \textbf{\textit{$Q_1$}} & \textbf{\textit{$\bar{x}$}} & \textbf{\textit{$Q_3$}} & \textbf{\textit{IQR}} & \textbf{\textit{\textcolor{red}{$\downarrow$} Atípico}} & \textbf{\textit{\textcolor{blue}{$\uparrow$} Atípico}} & \textbf{\textit{$\tilde{x}$}} & \textbf{\textit{$\hat{x}$}} & \textbf{\textit{$\sigma^2$}} & \textbf{\textit{$\sigma$}} \\ \hline
        \rowcolor{blue!6} Piso 1 & 24.06 & 25.07 & 26.05 & 1.99 & 22.08 & 28.06 & 24.98 & 25.45 & 2.44 & 1.56 \\ 
        Piso 2 & 19.7 & 20.34 & 21.0 & 1.3 & 18.38 & 22.29 & 20.28 & 20.17 & 0.86 & 0.93 \\ 
        \rowcolor{blue!6} Piso 3 & 19.14 & 20.2 & 21.68 & 2.54 & 16.39 & 24.01 & 19.57 & 19.21 & 3.41 & 1.85 \\ 
        Piso 4 & 21.03 & 21.93 & 22.63 & 1.6 & 19.52 & 24.33 & 21.92 & 21.15 & 2.35 & 1.53 \\ 
        \rowcolor{blue!6} Piso 5 & 24.56 & 25.75 & 26.68 & 2.12 & 22.56 & 28.94 & 25.72 & 25.15 & 3.09 & 1.76 \\ 
        Antena & 28.62 & 29.14 & 29.71 & 1.09 & 27.51 & 30.78 & 28.98 & 28.62 & 0.45 & 0.67 \\ 
        \rowcolor{blue!6} RF-VLC & 36.02 & \textcolor{blue}{$\uparrow$}36.9 & 37.15 & 0.53 & 36.1 & 37.69 & 36.81 & 36.02 & 0.23 & 0.48 \\ 
        Sótano 1 & 7.88 & 11.5 & 11.82 & 3.94 & 5.6 & 17.41 & 8.56 & 8.26 & 34.77 & 5.9 \\ 
        \rowcolor{blue!6} Sótano 2 & 4.61 & \textcolor{red}{$\downarrow$}4.93 & 5.18 & 0.57 & 4.07 & 5.78 & 4.93 & 4.57 & 0.12 & 0.34 \\ 
        \hline
    \end{tabular*}
    \label{tab:datos-estadisticos}
\end{table*}

Los datos recopilados se enfocaron principalmente en analizar la relación señal a ruido obtenida en distintos niveles y condiciones del Campus Universitario. Para garantizar la confiabilidad del análisis, se realizó previamente un escalado robusto, debido a la identificación inicial de valores atípicos; ya que al analizar el diagrama de violín ( Ver \Cref{Violinplot}), se observaron claramente valores atípicos significativos, especialmente en el Sótano 1. Este nivel presentó una dispersión notablemente mayor respecto a las otras ubicaciones evaluadas, indicando condiciones menos estables para la calidad del SNR.

%%%%%%%%%%%%%%% Figura chingona %%%%%%%%%%%%%%%     
    \begin{figure*}[b]
    \centering
    \centerline{\includegraphics[scale=0.4]{figures/Statistical Study/violin_plot.pdf}}
    \caption{Diagrama de Violín de Distribuciones SNR de una misma señal de radio frecuencia en diferentes espacios Indoor, muestreados cada 0.1 segundos.}
    \label{Violinplot}
    \end{figure*}    
%%%%%%%%%%%%%%% Figura chingona %%%%%%%%%%%%%%%
 
Estas anomalías en la distribución fueron cuantificadas mediante la prueba de \textbf{Shapiro-Wilk} \cite{ShapiroWilk1965}, que confirmó que los datos recolectados no siguen una distribución normal (\textbf{estadístico}=0.928,  \textbf{p-valor}<0.00). Esta evidencia respalda la elección de análisis estadísticos no paramétricos para las comparaciones posteriores.
Dado lo anterior, se aplicó la prueba \textbf{Kruskal-Wallis} \cite{Marusteri2010} para comparar los niveles de SNR entre los diferentes grupos evaluados. Los resultados mostraron una diferencia significativa (\textbf{estadístico}=1624.79, \textbf{p-valor}<0.00), indicando claramente que existen diferencias significativas en la calidad del SNR entre los diferentes niveles analizados.

%%%%%%%%%%%%%%% Figura chingona %%%%%%%%%%%%%%% 
\begin{figure*}[ht!]
    \centering
    \centerline{\includegraphics[scale=0.55]{figures/Statistical Study/scatterplot.pdf}}
    \caption{Diagrama de Dispersión SNR.}
    \label{Scatter}
\end{figure*}
%%%%%%%%%%%%%%% Figura chingona %%%%%%%%%%%%%%%
%%%%%%%%%%%%%%% Figura chingona %%%%%%%%%%%%%%%     
    \begin{figure*}[ht!]
    \centering
    \centerline{\includegraphics[scale=0.55]{figures/Statistical Study/histogram.pdf}}
    \caption{Histograma SNR.}
    \label{Histograma}
    \end{figure*}    
%%%%%%%%%%%%%%% Figura chingona %%%%%%%%%%%%%%% 

En la \cref{tab:datos-estadisticos}, la cual es el resumen estadístico que complementa visualmente estos hallazgos, resaltando especialmente la alta calidad del SNR en el sistema propuesto RF-VLC (media=36.9 dB) en contraste con la baja calidad observada en el "Sótano 2" (media=4.93 dB).
Finalmente, a pesar de la identificación clara de valores atípicos y la ausencia de una distribución normal, estos resultados respaldan positivamente la robustez y eficacia del sistema propuesto. Particularmente, se evidencia que bajo condiciones óptimas como la de "RF-VLC, el sistema supera considerablemente otros escenarios menos favorables, reafirmando el potencial para implementaciones prácticas. 

En la Figura \ref{Scatter}, las mediciones de dispersión evidencian en los sótanos, una clara tendencia descendente; indicando una degradación significativa de la señal en comparación con los otros niveles. Mientras que los pisos superiores presentan valores de SNR más estables, dado que los sótanos sufren mayores interferencias y absorción de señales, reflejando peores condiciones de transmisión. Una visión más gráfica se puede evidenciar en el histograma de la Figura \ref{Histograma}. Los peores valores de SNR se observan en los sótanos 1 y 2, indicando una mayor degradación de la señal en entornos subterráneos. En contraste, la terraza muestra los mejores valores de SNR, reflejando mejores condiciones de transmisión en comparación con los niveles inferiores.

% %%%%%%%%%%%%%%% Figura chingona %%%%%%%%%%%%%%% 
% \begin{figure*}[ht!]
%     \centering
%     \centerline{\includegraphics[scale=0.5]{figures/Statistical Study/scatterplot.pdf}}
%     \caption{Diagrama de Dispersión SNR.}
%     \label{Scatter}
% \end{figure*}
% %%%%%%%%%%%%%%% Figura chingona %%%%%%%%%%%%%%%
% %%%%%%%%%%%%%%% Figura chingona %%%%%%%%%%%%%%%     
%     \begin{figure*}[ht!]
%     \centering
%     \centerline{\includegraphics[scale=0.5]{figures/Statistical Study/histogram.pdf}}
%     \caption{Histograma SNR.}
%     \label{Histograma}
%     \end{figure*}    
% %%%%%%%%%%%%%%% Figura chingona %%%%%%%%%%%%%%% 

% %%%%%%%%%%%%%%%%%%%%%%%%%%%%%%%%%%%%%%%%%%%%%%%%%%%%%%%%%%%%% 
\clearpage
\subsection*{\textbf{Casos de Uso}} 
Ideas de innovación tecnológica.

%%%%%%%%%%%%%%% Figura chingona %%%%%%%%%%%%%%% 
    \begin{figure*}[b]
    \centerline{\includegraphics[scale=0.7]{figures/Casos de uso/ComuncacionSubacuatica.png}}
    % \caption{Comunicación Subacuática.}
        \caption{\textbf{Boya Inteligente RF-VLC para Estudios Marinos,} para facilitar investigaciones oceánicas sostenibles. Su antena RF permitiría comunicación constante con centros de investigación, mientras que el sistema VLC subacuático brindaría transferencia de datos rápida y segura a submarinos y vehículos no tripulados, impulsando estudios detallados sobre biodiversidad marina y cambios climáticos.}
    \vspace{0.5cm}
    \label{Comunicación Subacuática}
    \end{figure*}     
    
%%%%%%%%%%%%%%% Figura chingona %%%%%%%%%%%%%%%
 
\clearpage
% \subsection{Guiado Lumínico para Personas con Discapacidad Visual en Estaciones de Trenes} 
% Mediante señales lumínicas ubicadas estratégicamente en las estaciones de trenes, los usuarios en condiciones especiales de visión, equipados con dispositivos personales podrían orientarse, recibir información en tiempo real sobre rutas, plataformas y advertencias. Logrando así aumentar su autonomía e inclusión en espacios de transporte públicos. (Ver \Cref{Geolocalización para personas con discapacidad visual en estaciones de trenes})

%%%%%%%%%%%%%%% Figura chingona %%%%%%%%%%%%%%%
    \begin{figure*}[ht!]
    \centerline{\includegraphics[scale=0.8]{figures/Casos de uso/GeolocalizacionEnTrenesParaInvidente.jpg}}
    % \caption{Geolocalización para personas con discapacidad visual en estaciones de trenes}
    \caption{\textbf{Guiado Lumínico para Personas con Discapacidad Visual en Estaciones de Trenes,} mediante señales lumínicas ubicadas estratégicamente en las estaciones de trenes, los usuarios en condiciones especiales de visión, equipados con dispositivos personales podrían orientarse, recibir información en tiempo real sobre rutas, plataformas y advertencias. Logrando así aumentar su autonomía e inclusión en espacios de transporte públicos, (Ver \Cref{Geolocalización para personas con discapacidad visual en estaciones de trenes}).}
    \label{Geolocalización para personas con discapacidad visual en estaciones de trenes}
    \end{figure*}
% %%%%%%%%%%%%%%% Figura chingona %%%%%%%%%%%%%%% 

\clearpage

%%%%%%%%%%%%%% Figura chingona %%%%%%%%%%%%%%%%%%%%%%%%%%
    \begin{figure*}[htbp]
    \centerline{\includegraphics[scale=0.65]{figures/Casos de uso/MineriaSostenible.jpg}}
    % \caption{Minería Sostenible.}
    \caption{\textbf{Minería Sostenible,} en ambientes donde la seguridad y sostenibilidad son prioritarias, la combinación de RF y VLC asegura comunicaciones confiables y eficientes. Mineros equipados con cascos inteligentes pueden recibir información crítica a través de luces integradas en la infraestructura minera, mejorando la seguridad laboral, reduciendo accidentes y fomentando una operación minera sostenible y consciente del entorno. (Ver \Cref{Minería Sostenible}).}
    \label{Minería Sostenible}
    \end{figure*}
%%%%%%%%%%%%%% Figura chingona %%%%%%%%%%%%%%%%%%%%%%%%%%


\section{Conclusiones} \label{chap:ch5}

A través de una revisión detallada del estado del arte, se evidenció que tanto las tecnologías de radiofrecuencia como las de comunicación por luz visible presentan limitaciones particulares en entornos Indoor; sin embargo, sus características complementarias permiten una integración funcional. La tecnología de radiofrecuencias aporta cobertura y penetración, mientras que VLC ofrece alta velocidad y uso de espectro no licenciado. Esta compatibilidad teórica sirvió de base para la propuesta del sistema híbrido Radio-Óptico como solución viable.

Gracias al estado del arte, se logró diseñar una arquitectura funcional basada en el principio de heterodinaje, usando dispositivos USRP,  amplificadores de bajo ruido, mezcladores y componentes de acoplamiento como el Bias T y demás. La señal capturada en los 103.5 MHz fue transformada exitosamente a una frecuencia intermedia de 2 MHz; compatible con la respuesta en frecuencia de un LED blanco comercial usado, permitiendo su modulación para la posterior transmisión óptica, sin comprometer la integridad de la señal.

La implementación experimental del sistema híbrido demostró una transmisión funcional, empleando elementos optoelectrónicos y software definido por radio. Se verificó la modulación e iluminación del LED con la señal convertida y la detección efectiva por parte del fotodetector. Esto validó la operatividad básica del sistema y permitió su análisis bajo métricas técnicas relevantes como la SNR.

El análisis estadístico del sistema híbrido evidenció una mejora significativa en la relación señal a ruido, especialmente en comparación con la señal RF tradicional en entornos interiores desfavorables como sótanos. El sistema RF-VLC alcanzó una SNR promedio de 36.9 dB, muy superior a los valores de entre 4.9 y 11.5 dB en niveles subterráneos. Estas mejoras fueron consistentes en todas las pruebas, demostrando que la integración tecnológica impacta positivamente la calidad de la distribución de señal en ambientes urbanos Indoor.

Esta investigación evidencia una significativa mejora en la calidad y eficiencia de las telecomunicaciones en entornos interiores mediante la integración de tecnologías de radiofrecuencia y comunicación por luz visible. La convergencia tecnológica propuesta aborda soluciones de los sistemas RF, como la saturación del espectro, interferencias y limitaciones en entornos urbanos densamente poblados. La implementación experimental mostró claramente que la calidad de la señal, medida mediante la relación señal ruido, mejoró significativamente al aplicar este sistema híbrido, comparado al sistema exclusivamente RF.

El uso de VLC aprovecha significativamente el amplio ancho de banda disponible en el espectro visible, proporcionando velocidades de transmisión superiores sin necesidad de licencias, una ventaja considerable frente a los congestionados sistemas RF actuales. Este aspecto abre múltiples posibilidades de aplicaciones prácticas, tales como comunicación en escenarios subacuáticos, guiado lumínico para personas con discapacidad visual y minería sostenible. Escenarios que ejemplifican el potencial de esta tecnología para mejorar no solo la comunicación, sino también la seguridad y sostenibilidad en contextos críticos.

A nivel experimental, los resultados obtenidos revelan que la aplicación del sistema híbrido mejora sustancialmente la distribución uniforme de la señal, especialmente en lugares con condiciones desfavorables de propagación, como sótanos y edificaciones urbanas densas. Esto no solo valida la hipótesis inicial de la investigación, sino que además demuestra su aplicabilidad y potencial escalabilidad hacia múltiples ámbitos industriales y residenciales.

El sistema híbrido RF-VLC propuesto representa una solución efectiva y viable frente a los retos actuales de telecomunicaciones en interiores, además, abre camino a futuras investigaciones y desarrollos en comunicaciones ópticas integradas con sistemas de radiofrecuencia. Esta investigación tienen la capacidad de ser clave en la evolución de entornos conectados, seguros y eficientes, promoviendo así el desarrollo de ciudades inteligentes y sostenibles con infraestructuras de comunicación más robustas y accesibles.


\smallskip

\bibliographystyle{IEEEtran}
\bibliography{Referencias}

\end{document}
