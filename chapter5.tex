\section{Conclusiones} \label{chap:ch5}

A través de una revisión detallada del estado del arte, se evidenció que tanto las tecnologías de radiofrecuencia como las de comunicación por luz visible presentan limitaciones particulares en entornos Indoor; sin embargo, sus características complementarias permiten una integración funcional. La tecnología de radiofrecuencias aporta cobertura y penetración, mientras que VLC ofrece alta velocidad y uso de espectro no licenciado. Esta compatibilidad teórica sirvió de base para la propuesta del sistema híbrido Radio-Óptico como solución viable.

Gracias al estado del arte, se logró diseñar una arquitectura funcional basada en el principio de heterodinaje, usando dispositivos USRP,  amplificadores de bajo ruido, mezcladores y componentes de acoplamiento como el Bias T y demás. La señal capturada en los 103.5 MHz fue transformada exitosamente a una frecuencia intermedia de 2 MHz; compatible con la respuesta en frecuencia de un LED blanco comercial usado, permitiendo su modulación para la posterior transmisión óptica, sin comprometer la integridad de la señal.

La implementación experimental del sistema híbrido demostró una transmisión funcional, empleando elementos optoelectrónicos y software definido por radio. Se verificó la modulación e iluminación del LED con la señal convertida y la detección efectiva por parte del fotodetector. Esto validó la operatividad básica del sistema y permitió su análisis bajo métricas técnicas relevantes como la SNR.

El análisis estadístico del sistema híbrido evidenció una mejora significativa en la relación señal a ruido, especialmente en comparación con la señal RF tradicional en entornos interiores desfavorables como sótanos. El sistema RF-VLC alcanzó una SNR promedio de 36.9 dB, muy superior a los valores de entre 4.9 y 11.5 dB en niveles subterráneos. Estas mejoras fueron consistentes en todas las pruebas, demostrando que la integración tecnológica impacta positivamente la calidad de la distribución de señal en ambientes urbanos Indoor.

Esta investigación evidencia una significativa mejora en la calidad y eficiencia de las telecomunicaciones en entornos interiores mediante la integración de tecnologías de radiofrecuencia y comunicación por luz visible. La convergencia tecnológica propuesta aborda soluciones de los sistemas RF, como la saturación del espectro, interferencias y limitaciones en entornos urbanos densamente poblados. La implementación experimental mostró claramente que la calidad de la señal, medida mediante la relación señal ruido, mejoró significativamente al aplicar este sistema híbrido, comparado al sistema exclusivamente RF.

El uso de VLC aprovecha significativamente el amplio ancho de banda disponible en el espectro visible, proporcionando velocidades de transmisión superiores sin necesidad de licencias, una ventaja considerable frente a los congestionados sistemas RF actuales. Este aspecto abre múltiples posibilidades de aplicaciones prácticas, tales como comunicación en escenarios subacuáticos, guiado lumínico para personas con discapacidad visual y minería sostenible. Escenarios que ejemplifican el potencial de esta tecnología para mejorar no solo la comunicación, sino también la seguridad y sostenibilidad en contextos críticos.

A nivel experimental, los resultados obtenidos revelan que la aplicación del sistema híbrido mejora sustancialmente la distribución uniforme de la señal, especialmente en lugares con condiciones desfavorables de propagación, como sótanos y edificaciones urbanas densas. Esto no solo valida la hipótesis inicial de la investigación, sino que además demuestra su aplicabilidad y potencial escalabilidad hacia múltiples ámbitos industriales y residenciales.

El sistema híbrido RF-VLC propuesto representa una solución efectiva y viable frente a los retos actuales de telecomunicaciones en interiores, además, abre camino a futuras investigaciones y desarrollos en comunicaciones ópticas integradas con sistemas de radiofrecuencia. Esta investigación tienen la capacidad de ser clave en la evolución de entornos conectados, seguros y eficientes, promoviendo así el desarrollo de ciudades inteligentes y sostenibles con infraestructuras de comunicación más robustas y accesibles.
