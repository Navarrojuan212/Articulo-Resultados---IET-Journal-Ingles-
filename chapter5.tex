\section{Conclusions} \label{chap:ch5}

Through a detailed review of the state of the art, it became clear that both radio frequency and visible light communication technologies have particular limitations in indoor environments; however, their complementary characteristics allow for functional integration. Radio
frequency technology provides coverage and penetration, while VLC offers high speed and use of unlicensed spectrum. This theoretical compatibility served as the basis for the proposal of the Radio-Optical hybrid system as a viable solution. 
Thanks to state-of-the-art technology, it was possible to design a functional architecture based on the heterodyning principle, using USRP devices, low-noise amplifiers, mixers, and coupling components such as Bias T and others. The signal captured at 103.5 MHz was successfully transformed to an intermediate frequency of 2 MHz, compatible with the frequency response of a commercial white LED, allowing its modulation for subsequent optical transmission without compromising signal integrity.

The experimental implementation of the hybrid system demonstrated functional transmission using optoelectronic elements and radio-defined software. The modulation and illumination of the LED with the converted signal and the effective detection by the photodetector were verified. This validated the basic operability of the system and allowed its analysis under relevant technical metrics such as SNR.
Statistical analysis of the hybrid system showed a significant improvement in the signal-to-noise ratio, especially when compared to traditional RF signals in unfavorable indoor environments such as basements. The RF-VLC system achieved an average SNR of 36.9 dB, well above the values of between 4.9 and 11.5 dB at underground levels. These improvements were consistent across all tests, demonstrating that technological integration has a positive impact on the quality of signal distribution in indoor urban environments.

This research demonstrates a significant improvement in the quality and efficiency of telecommunications in indoor environments through the integration of radio frequency and visible light communication technologies. The proposed technological convergence addresses RF system solutions, such as spectrum saturation, interference, and limitations in densely populated urban environments. The experimental implementation clearly showed that signal quality, measured by the signal-to-noise ratio, improved significantly when applying this hybrid system, compared to the RF-only system. The use of VLC significantly exploits the ample bandwidth available in the visible spectrum, providing superior transmission speeds without the need for licenses, a considerable advantage over today's congested RF systems.
This aspect opens up multiple possibilities for practical applications, such as communication in underwater scenarios, light guidance for visually impaired people, and sustainable mining. 

These scenarios exemplify the potential of this technology to improve not only communication, but also safety and sustainability in critical contexts. At the experimental level, the results obtained reveal that the application of the hybrid system substantially improves the uniform distribution of the signal, especially in places with unfavorable propagation conditions, such as basements and dense urban buildings. This not only validates the initial research hypothesis, but also demonstrates its applicability and potential scalability to multiple industrial and residential settings.

The proposed RF-VLC hybrid system represents an effective and viable solution to the current challenges of indoor telecommunications. Furthermore, it paves the way for future research and development in optical communications integrated with radio frequency systems. This research has the potential to be key in the evolution of connected, secure, and efficient environments, thus promoting the development of smart and sustainable cities with more robust and accessible communication infrastructures.
%%%%%%%%%%%%%%%%%%%%
% La implementación experimental del sistema híbrido demostró una transmisión funcional, empleando elementos optoelectrónicos y software definido por radio. Se verificó la modulación e iluminación del LED con la señal convertida y la detección efectiva por parte del fotodetector. Esto validó la operatividad básica del sistema y permitió su análisis bajo métricas técnicas relevantes como la SNR.

% El análisis estadístico del sistema híbrido evidenció una mejora significativa en la relación señal a ruido, especialmente en comparación con la señal RF tradicional en entornos interiores desfavorables como sótanos. El sistema RF-VLC alcanzó una SNR promedio de 36.9 dB, muy superior a los valores de entre 4.9 y 11.5 dB en niveles subterráneos. Estas mejoras fueron consistentes en todas las pruebas, demostrando que la integración tecnológica impacta positivamente la calidad de la distribución de señal en ambientes urbanos Indoor.

% Esta investigación evidencia una significativa mejora en la calidad y eficiencia de las telecomunicaciones en entornos interiores mediante la integración de tecnologías de radiofrecuencia y comunicación por luz visible. La convergencia tecnológica propuesta aborda soluciones de los sistemas RF, como la saturación del espectro, interferencias y limitaciones en entornos urbanos densamente poblados. La implementación experimental mostró claramente que la calidad de la señal, medida mediante la relación señal ruido, mejoró significativamente al aplicar este sistema híbrido, comparado al sistema exclusivamente RF.

% El uso de VLC aprovecha significativamente el amplio ancho de banda disponible en el espectro visible, proporcionando velocidades de transmisión superiores sin necesidad de licencias, una ventaja considerable frente a los congestionados sistemas RF actuales. Este aspecto abre múltiples posibilidades de aplicaciones prácticas, tales como comunicación en escenarios subacuáticos, guiado lumínico para personas con discapacidad visual y minería sostenible. Escenarios que ejemplifican el potencial de esta tecnología para mejorar no solo la comunicación, sino también la seguridad y sostenibilidad en contextos críticos.

% A nivel experimental, los resultados obtenidos revelan que la aplicación del sistema híbrido mejora sustancialmente la distribución uniforme de la señal, especialmente en lugares con condiciones desfavorables de propagación, como sótanos y edificaciones urbanas densas. Esto no solo valida la hipótesis inicial de la investigación, sino que además demuestra su aplicabilidad y potencial escalabilidad hacia múltiples ámbitos industriales y residenciales.

% El sistema híbrido RF-VLC propuesto representa una solución efectiva y viable frente a los retos actuales de telecomunicaciones en interiores, además, abre camino a futuras investigaciones y desarrollos en comunicaciones ópticas integradas con sistemas de radiofrecuencia. Esta investigación tienen la capacidad de ser clave en la evolución de entornos conectados, seguros y eficientes, promoviendo así el desarrollo de ciudades inteligentes y sostenibles con infraestructuras de comunicación más robustas y accesibles.
