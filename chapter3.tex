\section{METHODOLOGY} \label{chap:ch3}

%%%%%%%%%%%%%%% Figura chingona %%%%%%%%%%%%%%%
\begin{figure*}[ht!]
\centerline{\includegraphics[scale=0.55]{figures/figure9.pdf}}
  \caption{Señal RF tomada desde el exterior y conectada al sistema VLC.}
  \label{Antenayagi}
\end{figure*} 
%%%%%%%%%%%%%%% Figura chingona %%%%%%%%%%%%%%% 

This chapter describes in detail the methodology used to achieve convergence between a radio broadcast link and a visible light communication system. The process begins with the capture of the commercial FM signal located at 103.5 MHz in the city of Medellin through a Yagi antenna installed on the fourth floor terrace of the ITM Fraternidad university campus (see diagram in \Cref{Antenayagi}). This signal was connected directly to a USRP, which, together with GNU Radio, allowed it to be processed and subsequently retransmitted at a frequency of 140 MHz.

\subsection{Integration of Broadcasting and Visible Light Communication}
At each stage of the circuit, measurements were taken using a TinySA portable reference spectrum analyzer to verify the performance of the system. As shown in \Cref{fig:Senal-In-Amplificador-Tx-140MHz}, a signal of -52.5 dBm without a DC component was recorded at the USRP output. After passing through the low-noise amplifier, this signal increased to -11 dBm, which can be clearly seen in \Cref{fig:Senal-Out-Amplificador-Tx-140MHz}. Simultaneously, using a second USRP, a 142 MHz signal was generated as the Local Oscillator frequency, which was combined with the 140 MHz signal using the AD831 mixer, resulting in two frequency components: 2 MHz (difference) and 282 MHz (sum). The 2 MHz component was chosen (see \Cref{fig:Senal-Out-Mixer-Tx-2MHz}) because it was suitable for the typical bandwidth of a conventional LED and simplified optical modulation. This signal, called the intermediate frequency (IF), had a power of -16.1 dBm, indicating an attenuation of approximately 5 dB compared to the previous stage. Finally, the signal was sent to a Bias-T circuit, which provided a DC level to excite the light source without interfering with the information. At the Bias-T output, the measured power was -17.1 dBm (see \Cref{fig:Senal-Out-BiasT-Tx-2MHz}), subsequently connecting to a 3W blue LED with yellow phosphor (obtained from a commercial LED flashlight), thus modulating its light intensity according to the electrical signal received.

% %%%%%%%%%%%%% Imagenes junstas super chingonas I %%%%%%%%%%%%%%%
\begin{figure*}[ht!]
    \centering
    % Primera fila
    \begin{subfigure}{0.45\textwidth}
        \centering
        \includegraphics[scale=0.6]{figures/figure10/figure10a.jpg}
        % \caption{Señal a la salida de la USRP, filtrada y retransmitida a 140 MHz.}
        \caption{Signal at the USRP output, filtered and retransmitted at 140 MHz.}
        \label{fig:Senal-In-Amplificador-Tx-140MHz}
    \end{subfigure}
    \hfill
        \begin{subfigure}{0.45\textwidth}
        \centering
        \includegraphics[scale=0.6]{figures/figure10/figure10b.jpg}
        % \caption{Señal después del amplificador LNA, utilizada para amplificar la señal recibida.}
        \caption{Signal after the LNA amplifier, used to amplify the received signal.}
        %\caption{Señal Out Amplificador Tx 140MHz.}
        \label{fig:Senal-Out-Amplificador-Tx-140MHz}
    \end{subfigure}
    
    % Segunda fila
    \vspace{0.5cm}
    \begin{subfigure}{0.45\textwidth}
        \centering
        \includegraphics[scale=0.6]{figures/figure10/figure10c.jpg}
        % \caption{Señal a la salida del mezclador, la cual corresponde a la señal de entrada al Bias-T.}
        \caption{Signal at the mixer output, which corresponds to the input signal to the Bias-T.}
        %\caption{Señal Out Mixer Tx 2MHz.}
        \label{fig:Senal-Out-Mixer-Tx-2MHz}
    \end{subfigure}
    \hfill
    \begin{subfigure}{0.45\textwidth}
        \centering
        \includegraphics[scale=0.6]{figures/figure10/figure10d.jpg}
        % \caption{Señal a la salida del Bias-T, que es la misma señal de entrada al LED.}
        \caption{Signal at the Bias-T output, which is the same as the LED input signal.}
        \label{fig:Senal-Out-BiasT-Tx-2MHz}
    \end{subfigure}
    \vspace{0.3cm}
    \caption{\textbf{Signals at Different Stages of the Hybrid Transmitter}, clearly showing how the signal power levels vary as they pass through the different stages of the hybrid transmitter. It starts with a power of approximately -52.5 dBm at the USRP output, passing through the amplification and frequency conversion processes, until it reaches a level of -17.1 dBm after the Bias-T, corresponding to the intermediate frequency.}

    \label{fig:ConjuntoSenales}
\end{figure*}

% \clearpage

During the reception stage (see \Cref{Rx-Hibrido}), the received signal was demodulated. For this purpose, a ThorLabs PDA25K(-EC) optical reference sensor with variable gain was used, which, according to the manufacturer's specifications, has an effective area of 4.8 mm² and operates in the range of 150 nm to 550 nm, with best sensitivity around 440 nm, within a temperature range between 0°C and 40°C. With a separation distance of 31 cm between the hybrid transmitter (Tx-Hybrid) and the hybrid receiver (Rx-Hybrid), the signal shown in  \Cref{fig:Senal-Rcibida-Sin-Lente-140MHz} was initially captured. It should be noted that this measurement was taken at the end of the Rx-Hybrid module, recording an approximate frequency of 140 MHz with a power of -53 dBm, as a reference for later evaluating the effect of incorporating a Fresnel lens to improve the SNR ratio. 

To analyze the impact of using the aforementioned Fresnel lens, detailed measurements were taken at each stage of the receiver. This lens It was located 13 cm from the ThorLabs optical sensor. It ThorLabs optical was captured sensor initially the optical signal at 2 MHz, after which the DC component was extracted using a Bias-T. As can be seen in \Cref{fig:Senal-In-Amplificador-Rx-2MHz}, the signal at this stage reached -63.1 dBm. Subsequently, the signal entered the low-noise amplifier, registering an output power of -33.6 dBm (see \Cref{fig:Senal-Out-Amplificador-Rx-2MHz}), consistent with the expected gain of the device (approximately 30 dB); the slight difference of 0.5 dBm could be attributed to losses in connections and cables. Continuing with the proposed scheme, the frequency change was performed using an AD831 mixer combined with a signal generated by a 142 MHz local oscillator, again obtaining a signal around 140 MHz. As shown in \Cref{fig:Senal-Rcibida-Lente-140MHz}, the recorded power was approximately -35 dBm, significantly better than the -53 dBm previously observed without the use of the Fresnel lens. This phase successfully concluded the integration of the proposed hybrid system. Additional details will will be presented in the results chapter. The next phase methodology will focus on the statistical analysis of the system.

%%%%%%%%%%%%%%% Imagenes junstas super chingonas II %%%%%%%%%%%%%%%%%
\begin{figure*}[b!]
    \centering
        \begin{subfigure}{0.45\textwidth}
            \centering
            \includegraphics[scale=0.6]{figures/figure11/figure11a.jpg}
            % \caption{Señal recibida en la etapa final de recepción sin usar lente de Fresnel.}
            \caption{Signal received in the final reception stage without using a Fresnel lens}
            \label{fig:Senal-Rcibida-Sin-Lente-140MHz}
        \end{subfigure}
    \hfill
        \begin{subfigure}{0.45\textwidth}
            \centering
            \includegraphics[scale=0.6]{figures/figure11/figure11b.jpg}
            % \caption{Señal medida a la salida del Bias T, ya sin componente DC usando lente de Fresnel.}
            \caption{Signal measured at the output of Bias T, now without DC component, using a Fresnel lens.}
            \label{fig:Senal-In-Amplificador-Rx-2MHz}
        \end{subfigure}

    %segunda fila
    \vspace{0.3cm}
        \begin{subfigure}{0.45\textwidth}
            \centering
            \includegraphics[scale=0.6]{figures/figure11/figure11c.jpg}
            % \caption{Señal a la salida del amplificador  de bajo ruido, usando lente de Fresnel.}
            \caption{Signal at the output of the low-noise amplifier, using a Fresnel lens.}
            \label{fig:Senal-Out-Amplificador-Rx-2MHz}
        \end{subfigure}
    \hfill
        \begin{subfigure}{0.45\textwidth}
            \centering
            \includegraphics[scale=0.6]{figures/figure11/figure11d.jpg}
            % \caption{Señal a la entrada del Doungle RTL2832U, usando lente de Fresnel.}
            \caption{Signal at the input of the RTL2832U dongle, using a Fresnel lens.}
            \label{fig:Senal-Rcibida-Lente-140MHz}
        \end{subfigure}
        % \vspace{0.5cm}
        \caption{\textbf{Signals in the receiver by stages.} Here, two clearly differentiated cases can be observed, considering a fixed distance between the transmitter and receiver of 134 cm. The first image corresponds to the signal measured at the end of the reception stage without using a Fresnel lens. The following three images show the measurements taken at different internal stages of the proposed hybrid receiver, including again the signal at the end of the receiver circuit, but now using a Fresnel lens placed at a distance of 13 cm from the photodetector. It should be noted that the total distance of the optical link remained constant in both cases.}
    \label{fig:ConjuntoSenales}
\end{figure*}

%%%%%%%%%%%%%%% Figura chingona %%%%%%%%%%%%%%% 
    \begin{figure*}[ht!]
    \centerline{\includegraphics[scale=0.45]{figures/figure12.png}}
    \caption{ITM University Institution, Fraternidad Campus.}
    % \caption{Institución Universitaria ITM, Campus Fraternidad.}
    \label{Campus Fraternidad}
    \end{figure*}
%%%%%%%%%%%%%%% Figura chingona %%%%%%%%%%%%%%% 

\subsection{Experimental Laboratory on SNR Degradation of an Urban Broadcasting Signal}\label{sec:se32}
In order to analyze how the signal-to-noise ratio (SNR) varies under different urban conditions, a study was conducted using the same commercial signal previously characterized in previous experiments, specifically at 103.5 MHz, within the urban context of Medellín. The analysis was carried out using a USRP device complemented by AIRSPY software. Strategic locations on the Fraternidad campus of the ITM University Institution (as shown in \Cref{Campus Fraternidad}) were chosen to take measurements, ranging from the basements (levels -1 and -2) to the terrace on the sixth floor, including the intermediate floors, which allowed for a comparative evaluation of the signal propagation conditions.

\clearpage
