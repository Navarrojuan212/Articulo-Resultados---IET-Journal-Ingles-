\section{STATE OF THE ART} \label{chap:ch2}
% \section{Estado del Arte} \label{chap:ch2}

The demand for wireless communications in indoor environments has increased significantly due to the growth of connected devices and the need for high data transfer rates \cite{Mitra2019,Chowdhury2019} [27], [28]. However, broadcasting and mobile communication services face multiple limitations in these scenarios, such as interference, signal attenuation, and radio spectrum saturation, resulting in poor performance of existing systems \cite{Lee2007,Tan2016} [29], [30]. \Cref{RF} shows a graphical representation of a real-world scenario of multiple signal broadcasting in an urban environment.

% \smallskip 

%%%%%%%%%%%%%%% Figura chingona %%%%%%%%%%%%%%%
\begin{figure*}[ht!]
% \begin{figure}[htbp]
\centerline{\includegraphics[scale=1.5]{figures/figure2.jpg}}
\caption{Radio Frequency Signals.}
\label{RF}
\end{figure*} 
%%%%%%%%%%%%%%% Figura chingona %%%%%%%%%%%%%%% 

These problems are evident in emerging technologies such as 5G and future 6G, and even in digital terrestrial television (DTT), the internet, and satellite television, or any communication service that involves a link from the outside to the inside, where the high frequencies used are susceptible to loss and require a line of sight free of obstacles \cite{Chaves2016,Ribadeneira-Ramírez,Guidotti2019} [31]–[33].

The migration from analog to digital systems and the use of white space spectrum have attempted to mitigate some of these problems, but significant challenges remain in spectral efficiency and spectrum management \cite{Wu2017,Zhang2017} [34], [35]. Given these limitations, it is necessary to explore alternatives that complement traditional radio frequency (RF) technologies and ensure reception and transmission quality inside any building, that is, maintaining the same level of communication quality whether in a home or in more inhospitable settings such as parking lots, shielded environments, tunnels, or underground mines.

VLC is a wireless communication technology that has been dominating scenarios such as indoor communication and could support RF systems \cite{Xu2016,Hao2013} [36], [37]. VLC uses the visible light band of the electromagnetic spectrum to transmit information, taking advantage of existing LED lighting infrastructure and offering benefits such as increased bandwidth, immunity to other radio signals, electromagnetic interference, and improved security \cite{Karunatilaka2015,RAHAIM2015T} [38], [39].

\smallskip 

However, VLC also has its own limitations, such as dependence on line of sight, interference with other light sources, and limited coverage \cite{Rahaim2015,Hosney2020} [40], [41]. Even so, VLC could be a technology that would allow the distribution of outdoor radio signals indoors, ensuring a more uniform signal-to-noise ratio in every corner of any building, especially in urban environments. To achieve this goal, an in-depth analysis of the architecture of RF and VLC communication systems is required to enable hybridization without interfering with the standards that currently govern these technologies.

This chapter analyzes the current limitations of RF and VLC technologies in indoor environments, with the aim of identifying opportunities to improve wireless communication in these spaces. \Cref{seccionRFLimitaciones} addresses the tech-RF technologies and their limitations, covering issues such as spectrum saturation, interference, and specific challenges in the implementation of 5G and 6G networks. \Cref{seccionVLC} focuses on VLC systems, addressing weaknesses such as their dependence on line of sight, optical interference, and challenges in modulation and bidirectional communication. \Cref{seccionConvergencia} proposes an architecture that links the radio signal capture system, the indoor distribution network, and VLC systems, and discusses the conditions and enabling technologies for hybrid RF-VLC systems and their possible application scenarios.

% \subsection{Tecnología RF y sus Limitaciones en Entornos Indoor}
\subsection{RF Technology and Its Limitations in Indoor Environments}
\label{seccionRFLimitaciones}

    RF technologies in indoor environments can experience significant attenuation due to absorption by walls and other obstacles, which is especially problematic at higher frequencies such as the millimeter waves used in 5G \cite{Singh2017} [42]. Attenuation increases in densely populated areas, where multiple reflections and scattering further complicate signal propagation \cite{Echarri} [43]. In these environments where there is human movement, signal blocking by bodies and scattering by objects affect the quality of signal propagation in 28 GHz links \cite{Dalveren2019,Benzaghta2021} [44], [45]. In addition, the coexistence of multiple wireless devices and systems generates electromagnetic interference that can affect network performance, increasing latency and the likelihood of data loss \cite{Chiaramello2019} [46].

    On the other hand, saturation in Internet of Things (IoT) devices will continue to grow. As applications in environments such as smart cities, digital health, and industry continue to expand, IoT systems are forced to handle large volumes of data, facing various limitations, such as energy consumption and data transmission reliability  \cite{Ebenezer2023} [47]. These factors have led to significant saturation, to the point of a shortage of available spectrum, especially in unlicensed bands such as ISM 2.4GHz \cite{Hou2022} [48]. To address these challenges related to signal attenuation and spectrum saturation, it is essential to understand how the radio spectrum is distributed and used in different frequency bands. Below (see \Cref{tabla1}) are some of the technologies that operate in each of these bands. This information allows us to visualize the areas of greatest congestion and the opportunities for implementing new developments that optimize the performance and management of wireless networks.

    
    \begin{table}[ht!]
    \caption{Tecnologías RF en Diferentes Bandas del Espectro}
    \begin{center}
    \begin{tabular}{cccc}
    \hline
    \textbf{\textit{Tecnología}} & \textbf{\textit{Frec (GHz)}}& \textbf{\textit{Saturación}}& \textbf{\textit{Ref}} \\
    \hline
    4G LTE, 5G   & 0.8 - 2.6  & Alta & \cite{Kim2019,Lee2022,Qian2020} \\ 
    WiFi (802.11b/g/n)  &  2.4  & Muy alta  &  \cite{Nasser2021,Ahmad2020}   \\
    5G (banda C) &  3.5 & En aumento & \cite{Lee2018,Hu2018,Frieden2020}  \\
    WiFi (802.11a/ac/ax)   &  5 & Alta & \cite{Abid2023,Avallone2021,Chen2020}  \\
    5G (mmWave) & 24-28 & Baja & \cite{Singh2019,Sung2020,Erofeev2020} \\
    WiGig & 60 & Baja & \cite{Fierro2020,Liu2022,Kim2021} \\
    Punto a punto & 70-80 & Muy Baja & \cite{Larsson2020,Liu2021} \\
    \hline
    \end{tabular}
    \label{tabla1}
    \end{center}
    \end{table}


    To overcome some of the limitations of conventional wireless technologies, it is necessary to explore alternatives in different regions of the electromagnetic spectrum in order to reduce interference and increase efficiency in indoor environments. In the following section, we will discuss visible light communication in detail, analyzing its benefits, limitations, and potential integration with other technologies. to optimize performance in indoor environments.

%%%%%%%%%%%%%%%%%%%%%%%%%%%COPIA TABLA%%%%%%%%%%%%%%%%%%%%%%%%%%%%%%%%%%%%%%%%%%%%%%%%%%%%%%
    \begin{table*}[ht!]
    \caption{Algunos dispositivos VLC y sus posibles aplicaciones.}
    \centering
    % \renewcommand{\arraystretch}{1.3}
    % \begin{tabular}{|c|c|c|>{\raggedright\arraybackslash}p{9cm}|}
    \begin{tabularx}{\textwidth}{cXXc}
    \hline
    \textbf{\textit{Tipo de Tx y/o Rx }} & \textbf{\textit{Descripción}} & \textbf{\textit{Aplicación }} & \textbf{\textit{Ref}} 
    \\ \hline
    RGB LED PAM-4  & Emplea LEDs tricolores para transmitir datos mediante modulación PAM-4. & Sistemas de alta velocidad con baja interferencia. & \cite{Wang2020,Zhou2019}
    \\ \hline
    LED-to-LED VLC  & Se utiliza un LED tanto transmisor como receptor. Esto permite una comunicación de bajo costo. & Aplicaciones de corto alcance y  bajo costo. & \cite{Mir2021,Pramudya2023}  
    \\ \hline
    \makecell{Silicon Photomultiplier \\(SiPM) Receiver}  & Receptor con SiPM que opera con longitudes de onda de 405 nm. Con alta tasa de transmisión de datos y puede soportar hasta 1 Gbps. & Aplicaciones Indoor, sistemas de comunicación rápidos. & \cite{Ali2020,Ma2023} 
    \\ \hline
    \makecell{Fotodetector con\\ lentes Fresnel} & Para mejorar la ganancia óptica. Ofrece baja latencia y transmisión robusta. & Sistemas de transporte inteligente y comunicación vehículo-infraestructura. & \cite{Nawaz2020,Younus2023} 
    \\ \hline
    \makecell{Transmisores LED \\con MOSFET} & Emplea LED y MOSFET como amplificador sumador. Se usa para transmisión en distancias variables con codificación Manchester.& Comunicación vehículo a vehículo e infraestructura. & \cite{Huang2021,Greives2020} 
    \\ \hline
    \makecell{Perovskite Dual-Band \\Photodetector} & Utiliza fotodetectores de perovskita con capacidad de respuesta dual. Puede captar señales de diferentes longitudes de onda. & Comunicación eficiente y de alta velocidad Indoor & \cite{Huang2020} 
    \\ \hline
    \end{tabularx}
    \label{tabla2}
    \end{table*}
    %%%%%%%%%%%%%%%%%%%%%%%%%%%%%%%%%%%%%%%%%%%%%%%%%%%%%%%%%%%%%%%%%%%%%%%%%%%%%%%%%%%%%%

% \subsection{Sistemas de Comunicación por Luz Visible }
    \subsection{Visible Light Communication Systems}
    \label{seccionVLC}

    %%%%%%%%%%%%%%% Figura chingona %%%%%%%%%%%%%%% 
    \begin{figure*}[ht!]
    \centerline{\includegraphics[scale=0.2]{figures/figure3.png}}
    % \caption{Esquema general de un sistema de comunicación por luz visible.}
    \caption{General diagram of a visible light communication system.}
    \label{Esquema General VLC}
    \end{figure*}
    %%%%%%%%%%%%%%% Figura chingona %%%%%%%%%%%%%%% 

    Unlike RF signals, which operate in a regulated and licensed spectrum range, mainly between 3 kHz and 300 GHz, VLC technology operates in the visible range of 400 to 800 THz, a part of the spectrum considerably higher than that allocated to radio communications; this difference in frequency range allows both technologies to coexist without interference. Visible light communication, in addition to being license-free \cite{Mushfique2018} [79], offers benefits such as greater energy efficiency and a greater capacity to transmit data at high speeds \cite{Chi2020} [80]. Compared to RF, VLC uses energy more efficiently, as it can take advantage of existing LED lighting infrastructure, allowing data transmission with almost no additional energy consumption while simultaneously providing illumination \cite{Chamani2022,Nawawi2019} [81], [82]. The transmitters used in VLC systems are LEDs, which can be modulated at high speeds for data transmission \cite{Farid2022,Kong2018} [83], [84], while the receivers use photodetectors such as photodiodes \cite{Bishnu2018} [85] or silicon photomultipliers (SiPM) \cite{Ahmed2019}  [86], designed to capture modulated light and decode information. \Cref{tabla2} presents some of these devices for visible light communication along with their possible technical applications.

    % \subsection{ Tecnología LED }
    \paragraph{LED technology}
    Light-emitting diodes have transformed modern lighting thanks to their energy efficiency and durability. Physically, they are PN-type junction semiconductor devices, where the P junction is doped with acceptor atoms and the N junction is doped with donor atoms, which, when directly polarized, allow current to flow in one direction through the semiconductor material, causing radioactive recombination of electrons and holes in the depletion region; This causes photons to be emitted in the form of light \cite{Cengiz2022} [87]. LEDs can be classified according to the semiconductor material used \cite{Rahman2014} \cite{Marciniak2019} [88] [89] or their specific application. Among the most common types are infrared LEDs, used mainly in remote controls and proximity sensors. On the other hand, red, green, and blue (RGB) LEDs have been fundamental in the development of displays \cite{Wang2022} [90] and adjustable lighting systems, where the mixture of these colors allows for a wide range of colors \cite{Muthu2002} [91]. The emission color of LEDs is related to semiconductor materials with which they are manufactured; \Cref{LED Semiconductor Materials} provides a brief overview of this.
    

    \begin{table}[ht!]
        \centering
        \caption{Materiales Semiconductores de los LEDs, tomado de \cite{Ghassemlooy2017}.}
        \begin{tabular}{lcl}
        \toprule
        \textbf{Color}       & \textbf{$\lambda$ (nm)} & \textbf{Material Semiconductor}                                \\
        \midrule
        \addlinespace[3pt]
        
        Infrarrojo             & $\lambda > 760$           & GaAs, AlGaAs                                                     \\
        Rojo                  & $610 < \lambda < 760$     & AlGaAs, GaAsP, AlGaInP, GaP                                       \\
        Naranja               & $590 < \lambda < 610$     & GaAsP, AlGaInP, GaP                                               \\
        Amarillo               & $570 < \lambda < 590$     & GaAsP, AlGaInP, GaP                                               \\
        Verde                & $500 < \lambda < 570$     & InGaN/GaN, GaP, AlGaInP, AlGaP                                    \\
        Azul                 & $450 < \lambda < 500$     & ZnSe, InGaN                                                       \\
        Violeta               & $400 < \lambda < 450$     & InGaN                                                             \\
        Ultravioleta          & $\lambda < 400$           & Diamante, AlGaN, AlGaInN                                           \\
        \bottomrule
        \end{tabular}
        \label{LED Semiconductor Materials}
    \end{table}

    %%%%%%%%%%%%%%%%%%%%%

    White LEDs, essential in general lighting, can be obtained using two main methods: combining RGB LEDs or using blue LEDs coated with phosphor. In the first case, white light is generated by mixing three individual primary color LEDs (RGB), allowing f o r precise adjustments in color reproduction. In the second method, a blue LED excites a layer of yellow phosphor, generating highly efficient, low-cost white light, which has become the dominant technology in the lighting market. 
    
    Other relevant types include ultraviolet LEDs, widely used in sterilization processes, substance detection, and material curing in industrial and medical applications. Finally, high-brightness and high-power LEDs have been designed for demanding applications, such as architectural lighting and the automotive industry, where high light intensity and high resistance to adverse environmental conditions are required. The response of LEDs is presented below in general terms, given that it is key to consider that in any VLC system, the transmission process will depend directly on this characteristic.

    % \subsection{ Respuesta en Frecuencia del LED}
    \paragraph{LED Frequency Response}
    White light can be reproduced using LED technology, either through a combination of RGB LED luminaires or blue LEDs with one or more layers of phosphor. Commercially, the most common are blue LED bulbs with a layer of yellow phosphor \cite{Li2022} [93], which is used to convert the blue light emitted by the LED chip into white light. This is achieved by emitting yellow light which, when combined with blue light, produces high-quality white light. These LEDs have the advantage of being cheaper, easier to manufacture, and having good luminous efficiency, offering a simple and efficient way to generate white light, albeit with a slight reduction in the color rendering index \cite{Chen2019} [94]. 

    A frequency response of around 2 MHz is commonly observed in white LEDs based on blue LED technology with yellow phosphor, as reported in various studies \cite{Chen2019} [94]. This behavior is mainly due to the decay time of the yellow phosphor, which is on the order of 80 ns, resulting in a 3 dB bandwidth close to 2 MHz. This parameter is crucial in determining the modulation capability of these devices for indoor visible communications applications \cite{Wang2024,Park2004,Sun2015} [95]–[97]. Furthermore, the modulation offered by these LEDs is perfectly aligned with the speed and light quality requirements necessary for commercial visible communications systems, while simultaneously enabling high light efficiency \cite{Hu2015} [98] and a good color rendering index \cite{Zhang2018} [99].

    %%%%%%%%%%%%%%% Figura chingona %%%%%%%%%%%%%%% 

    \begin{figure*}[ht!]
    \centerline{\includegraphics[scale=0.5]{figures/figure4.jpg}}
    \caption{LOS, NLOS, Diffuse Link, Quasi-diffuse. Estudio tomado de  \cite{Rahman2018}}
    \label{LOS_NLOS}
    \end{figure*}
    
    %%%%%%%%%%%%%%% Figura chingona %%%%%%%%%%%%%%% 
    
    
    % \subsection{Dependencia de LOS y Desafíos de Interferencia en Sistemas VLC}
    \paragraph{Dependence on LOS and Interference Challenges in VLC Systems}
    Visible light communication systems rely heavily on Line of Sight (LOS) to achieve optimal performance between the transmitting LED and the receiver. This is key, as diffuse or non-directional configurations limit the achievable rate. LOS increases the intensity of the received signal and implies a higher SNR, which allows for an increase in the data rate or link distance and reduces the risk of Inter-Symbol Interference (ISI) \cite{Ghassemlooy2017} [92]. When obstructions are present, communication links can suffer a significant drop in performance. This LOS dependence has motivated studies to improve blocking tolerance, either through reflective surfaces or beam angle optimization technologies \cite{Cao2019,Hamad2023} [101], [102].
    
    % %%%%%%%%%%%%%%% Figura chingona %%%%%%%%%%%%%%% 
    In \Cref{LOS_NLOS} Figure 4, the authors mention four possible configurations (LOS, NLOS, Diffuse, Quasi-Diffuse) which are key to channel modeling and a correct description of a VLC system \cite{Rahman2018} [100]. \Cref{tabla3} summarizes the main issues previously mentioned in visible light communication systems. To mitigate the negative effects of channel distortion and inter-symbol interference in communication systems, various compensation techniques have been developed. These techniques seek to maintain link quality and stability, improving system performance. Next, equalization will be addressed as one of the techniques relevant to our research.


    \begin{table}[ht!]
    \caption{Challenges in VLC Links}
    \begin{center}
    \begin{tabular}{ccc}
    \hline
    \textbf{\textit{Desafío}} & \textbf{\textit{Descripción}}& \textbf{\textit{Ref}} \\ 
    \hline
    \makecell{Dependencia\\ de LoS}    & \makecell{VLC requiere una trayectoria \\sin obstrucciones entre el\\ transmisor y el receptor, \\lo que limita su uso en \\ambientes con obstáculos. } & \cite{Rodoplu2020,Zhou2023} \\ 
    \hline
    \makecell{Interferencia \\Óptica}  & \makecell{La luz ambiental y fuentes de luz\\ artificial causan interferencia \\en los enlaces VLC, afectando \\la claridad de la señal. } & \cite{Forkel2019,affan2020}   \\
    \hline
    \makecell{Desafíos en \\Comunicación \\Bidireccional}  & \makecell{El control bidireccional en VLC es\\ complejo debido a la alineación \\estricta de los dispositivos transmisores\\ y receptores para mantener la LOS. } &\cite{Almohanna2019,Wei2018}  \\
    \hline
    \end{tabular}
    \label{tabla3}
    \end{center}
    \end{table}
    
% \subsection{Técnicas de Compensación de Errores}
\paragraph{Error Compensation Techniques}
Error compensation techniques are fundamental in communications systems, as they allow for the mitigation of adverse effects of the transmission channel, such as noise, dispersion, and interference \cite{Djordjevic2022} [109], thus ensuring integrity of the information received. Among the various existing techniques, the following are particularly noteworthy: channel error correction (Error Correction Coding, ECC) \cite{Mambou2021} [110] and equalization \cite{Metzger2011} [111]. Where ECC adds redundancy to the transmitted data, allowing errors to be detected and corrected in the receiver, while equalization combats distortion by adjusting the signal to counteract unwanted effects \cite{Nguyen2019} [112].

In optical communications, and especially in VLC, error compensation is particularly important due to the variable and unpredictable nature of the optical channel \cite{Mambou2021} [110], which is affected by multipath dispersion, noise, and the limitations of the transmitting and receiving devices \cite{Shapiro2023} [113]. For this reason, pre-equalization and post-equalization techniques are used, and increasingly, hybrid methods that combine both strategies \cite{Cenqin2021} [114].

% \subsection{Ecualización}
\paragraph{Equalization}

Equalization is an essential technique in communication systems, whose objective is to counteract the distortions and limitations imposed by the transmission channel. These distortions may be due to the limited frequency response of the components, temporal dispersion, the non-linearity of the devices (such as LEDs in VLC), and interference between symbols. Equalization improves the quality of the received signal, increases the data transmission rate, and reduce the bit error rate (BER). There are different equalization strategies: pre-equalization, post-equalization, and combined schemes, each with specific advantages and limitations. The choice of the appropriate technique depends on the characteristics of the channel, the complexity allowed in the transmitter and receiver, and the performance requirements of the system.

The combination of pre- and post-equalization can offer additional improvements in system robustness and efficiency, especially in channels with high dispersion or nonlinearity \cite{Siuzdak2022,Singh2022} [115], [116].

% %%%%%%%%%%%%%%% Figura chingona %%%%%%%%%%%%%%% 

    \begin{figure*}[ht!]
    \centerline{\includegraphics[scale=0.34]{figures/RF-VLC/figure5.pdf}}
    % \caption{Amplificador RF y Filtro pasabanda en la entrada del receptor. Tomado de \cite{Carr2000}}
    \caption{Receptor Heterodineo. Tomado de \cite{Carr2000}.}
    \label{Heterorinaje-RF}
    \end{figure*}
    
%     %%%%%%%%%%%%%%% Figura chingona %%%%%%%%%%%%%%% 

% \subsubsection{Pre-ecualización \& Pos-ecualización }
% \paragraph{Pre-ecualización \& Pos-ecualización }
\paragraph{Pre-equalization \& Post-equalization}

Pre-equalization is implemented in the transmitter and consists of modifying the signal before sending it, anticipating and compensating for foreseeable channel distortions. In optical and VLC systems, it is especially useful for counteracting the limited bandwidth and non-linearity of LEDs, thus allowing higher transmission rates and reducing complexity in the receiver \cite{Surampudi2022} [118]. Digital pre-equalization schemes have been shown to increase effective bandwidth and improve noise tolerance, achieving significant increases in data rate without the need for post-equalization \cite{Li2022} [93]. However, their effectiveness depends on precise knowledge of the channel and may be limited by the capacity of the pre-equalization circuits, especially in analog implementations \cite{Yan2020} [119].

On the other hand, post-equalization is applied in the receiver and uses adaptive filters or advanced algorithms to correct distortions affecting the received signal, such as ISI caused by temporal or modal dispersion. It is particularly useful when the channel is variable or difficult to model in the transmitter, although it can amplify noise and increase the complexity of the receiver \cite{Ge2020} [120]. In highly nonlinear channels or channels with a high signal-to-noise ratio, post-equalization can outperform pre-equalization in terms of received signal quality. Furthermore, the combination of both techniques (pre- and post-equalization) has been shown to improve the overall performance of the system, reducing the power required and improving the sensitivity of the receiver \cite{Siuzdak2019} [121].

% \subsubsection{Técnicas híbridas y tendencias recientes}
\paragraph{Hybrid techniques and recent trends}
Currently, VLC research is exploring hybrid approaches that combine pre-equalization and post-equalization, as well as advanced machine learning-based algorithms, to achieve greater robustness and flexibility \cite{LiangQiao2018,Chi2018} [122], [123]. For example, neural networks and deep learning are being applied to design adaptive equalizers that can significantly improve performance in changing environments \cite{Guan2018,Lu2025,Li2023} [124]–[126]. Furthermore, the trend toward integrating VLC with other technologies such as RF (e.g., hybrid RF-VLC systems) makes the study of adaptive and intelligent error compensation techniques even more relevant \cite{Sun2023,daSilva2024,Abdallah2025} [127]–[129]. In other words, the efficient treatment of errors in VLC systems depends on the appropriate combination of pre-equalization, post-equalization, and hybrid techniques, adjusted to the environment and the nature of the optical channel, which is crucial for the viability of robust RF-VLC hybrid systems.
    
% \newpage
\subsection{Convergence between RF \& VLC Systems}
    \label{seccionConvergencia}

 The convergence between radio frequency and visible light communication technologies is presented here not as a generic solution, but as a specific response to the need to capture an RF signal in an open environment and retransmit it efficiently in a confined space through an optical optical system. This proposal, which combines system Reception by antenna with retransmission using modulated light is based on principles widely described in the literature on superheterodyne receivers and the generalities of indoor VLC systems (see Figure 5 and Figure 6, respectively), where the use of mixers and local oscillators allows the signal frequency to be adapted for subsequent processing without loss of fidelity \cite{Carr2000} [117]. By integrating these elements with a VLC optical channel, the aim is to Take advantage of low interference, high security, and the growing availability of indoor LED infrastructure to operate in indoor spaces \cite{Karunatilaka2015} [38].

Additionally, to minimize errors and facilitate component integration, the use of pre-assembled modules is proposed, such as mixers, low-noise amplifiers (LNA), and Bias-T coupling circuits, which ensure the operational stability of the system during testing. This research is based on the premise that it is possible not only to combine the advantages of both technologies, but also to design a continuous and functional signal flow that maintains a uniform SNR even in physically hostile environments. The technical description of the heterodyning process and the detailed implementation of the hybrid system are developed in the following subsection.

    %%%%%%%%%%%%%%% Figura chingona %%%%%%%%%%%%%%% 

    \begin{figure*}[ht!]
    \centerline{\includegraphics[scale=0.35]{figures/RF-VLC/figure6.pdf}}
    \caption{Generalidad de los Sistemas VLC-Indoor \cite{Karunatilaka2015}.}
    \label{VLC-Indoor}
    \vspace{10pt} % Espacio opcional entre la leyenda y la línea
    \rule{\textwidth}{0.2pt} % Mismo ancho que la imagen
    \vspace{8pt} % Espacio opcional entre la leyenda y la línea
    \end{figure*}
    
    %%%%%%%%%%%%%%% Figura chingona %%%%%%%%%%%%%%% 
    
% \subsection{Sistema Híbrido Propuesto}
\paragraph{Proposed Hybrid System}
This paper proposes a hybrid communications system that integrates RF technology with a VLC system. The approach is based on a review of the specialized literature and the various studies found, which highlight the complementarity between both domains for scenarios with high spectral demand and localized security \cite{BravoAlvarez2023} [130]. Inspired by these approaches, a design is proposed that takes advantage of the penetration and robustness of RF, together with the directionality, physical security, and wide free spectrum offered by VLC \cite{Wang2023} [131]. The system consists of two main stages: the first stage is the capture and processing of RF signals for conversion to the optical domain; and the second stage is responsible for receiving and demodulating the VLC signal to recover the base information. The range specifications and characteristics of the equipment are presented in \Cref{Componentes del Transmisor Híbrido.} \& \Cref{Componentes del Receptor Híbrido}.
% \smallskip
\paragraph{Transmission Stage of the RF–VLC Hybrid System}
In this stage, the RF signal is picked up by an antenna dipole coupled to a USRP 2900; which provides the necessary flexibility for signal processing. This signal is then connected to a low-noise amplification block; it is then fed to an AD831 mixer, which, in conjunction with a local oscillator,
implements RF heterodyning, shifting the signal to a desired intermediate frequency (IF). It is then coupled to the optical emitter via a bias tee, which is used to apply a DC signal necessary to polarize and turn on the LED, which is used as an optical modulator in an IM/DD scheme to emit the signal in
the visible light domain \cite{Yeh2015} [132]. 

    %%%%%%%%%%%%%%% Figura chingona %%%%%%%%%%%%%%% 
    % \begin{figure*}[ht!]
    \begin{figure*}[ht!]
    \centerline{\includegraphics[scale=0.565]{figures/RF-VLC/figure7.jpg}}
    \caption{Transmisor Híbrido Propuesto.}
    \label{Tx-Hibrido}
    % \vspace{10pt} % Espacio opcional entre la leyenda y la línea
    \rule{\textwidth}{0.2pt} % Mismo ancho que la imagen
    \vspace{20pt} % Espacio opcional entre la leyenda y la línea
    \end{figure*}
    %%%%%%%%%%%%%%% Figura chingona %%%%%%%%%%%%%%% 
    %%%%%%%%%%%%%%% Figura chingona %%%%%%%%%%%%%%% 
    \begin{figure*}[ht!]
        \centerline{\includegraphics[scale=0.5]{figures/RF-VLC/figure8.jpg}}
        \caption{Rx Híbrido.}
        \label{Rx-Hibrido}
        % \vspace{10pt} % Espacio opcional entre la leyenda y la línea
        % \rule{\textwidth}{0.2pt} % Mismo ancho que la imagen
        % \vspace{22pt} % Espacio opcional entre la leyenda y la línea
    \end{figure*}
%%%%%%%%%%%%%%% Figura chingona %%%%%%%%%%%%%%% 
% \smallskip
% \subsubsection{Etapa de Recepción del Sistema Híbrido RF-VLC} 
\paragraph{RF-VLC Hybrid System Reception Stage} 
The optical signal emitted by the LED is captured by a ThorLabs PDA25K(-EC) sensor, chosen for its high sensitivity in the visible range. In this reverse process, it is necessary to extract the DC component of the signal, which is done using a Bias Tee (with the same technical specifications as the transmission stage). The signal is then amplified with an LNA module identical to the one used in the transmission stage to maintain symmetry in the processing chain. 
Next, the previous block is connected to an AD831 mixer which, together with a USRP 2901 as a local oscillator, performs the inverse frequency conversion, allowing the original base signal to be recovered. Finally, the recovered signal is connected to an RTL2832U, a device widely used for reception testing in various RF bands. This design is based on the use of pre-assembled electronic components, whose benefits include accelerated construction and experimental validation, high compatibility with open-source software platforms, and a reliable basis for academic experimentation.
The proposed hybrid system offers a functional and reproducible solution that articulates the conversion of RF signals to the VLC domain and their subsequent recovery, opening the door to new experimental applications. The technical details can be seen in Table VII. The following section describes the entire methodological process of technological integration. 

% %%%%%%%%%%%%%%% Figura chingona %%%%%%%%%%%%%%% 
%     \begin{figure*}[ht!]
%         \centerline{\includegraphics[scale=0.6]{figures/RF-VLC/figure8.jpg}}
%         \caption{Rx Híbrido.}
%         \label{Rx-Hibrido}
%     \end{figure*}
% %%%%%%%%%%%%%%% Figura chingona %%%%%%%%%%%%%%% 

%%% OTRA VERSION DE TABLAS
\begin{table*}[ht!]
    \centering
    \footnotesize
    \renewcommand{\arraystretch}{1.3}

    %------------ TABLA IZQUIERDA (TX) -----------------
    \begin{minipage}[t]{0.48\textwidth}
        \caption{\textbf{Componentes del Transmisor Híbrido.}}
        \label{tab:tx_hibrido}
        \centering
        \begin{tabular}{@{}p{3cm}>{\raggedright\arraybackslash}p{5cm}@{}}
        \hline
        \textbf{\textit{Dispositivo}} & \textbf{\textit{Descripción}} \\ 
        \hline
        Low-Noise Amplifier  & Amplificador RF banda ancha de alta ganancia (30\,dB), 
        con rango de operación de 0,1--2000\,MHz. \\
        \hline
        Mixer - AD831 & Mezclador de un solo chip de amplio rango dinámico y baja distorsión, 
        opera entre 0,1--500\,MHz con 10\,dBm de ganancia, con condensadores de aislamiento DC. \\
        \hline
        Bias Tee & RF Microondas HF Bias Tee DC Bias 50K--60\,MHz. Fuente de alimentación de antena activa. \\
        \hline
        LED Azul con Fósforo Amarillo & Linterna de luz blanca convencional con una respuesta en frecuencia típica de 2\,MHz. \\
        \hline
        USRP 2900, 2901 & Del fabricante National Instruments con un rango de operación RF 
        entre los 70\,MHz--6\,GHz. \\
        \hline
        \end{tabular}
    \end{minipage}
    \hfill
    %------------ TABLA DERECHA (RX) -----------------
    \begin{minipage}[t]{0.48\textwidth}
        \caption{\textbf{Componentes del Receptor Híbrido.}}
        \label{tab:rx_hibrido}
        \centering
        \begin{tabular}{@{}p{3cm}>{\raggedright\arraybackslash}p{5.7cm}@{}}
        \hline
        \textbf{\textit{Dispositivo}} & \textbf{\textit{Descripción}} \\ 
        \hline
        Sensor Óptico ThorLabs & Del fabricante ThorLabs con referencia PDA25K(-EC) y ganancia ajustable.\\
        \hline
        Bias Tee & RF Microondas HF Bias Tee DC Bias 50K--60\,MHz. Fuente de alimentación de antena activa.\\
        \hline
        Low-Noise Amplifier & Amplificador RF banda ancha de alta ganancia (30\,dB), 
        con rango de operación de 0,1--2000\,MHz.\\
        \hline
        Mixer AD831 & Mezclador de un solo chip de amplio rango dinámico y baja distorsión, 
        opera entre 0,1--500\,MHz con 10\,dBm de ganancia, con condensadores de aislamiento DC.\\
        \hline
        RTL2832U & El RTL2832U es un chip popular utilizado en dongles de software-defined radio (SDR).\\
        \hline
        USRP 2900 & Del fabricante National Instruments con un rango de operación RF 
        entre los 70\,MHz--6\,GHz.\\
        \hline
        \end{tabular}
    \end{minipage}
\end{table*}

%%%%%%%%%%%%%%%%%%%%%%%%%%%%%%%%%%%%%%%%%%%%%%

% \begin{table*}[ht!]
%     \caption{\textbf{Componentes del Transmisor Híbrido.}}
%     \centering
%     \footnotesize
%     \renewcommand{\arraystretch}{1.3}
%     \begin{tabular}{@{}p{3cm}>{\raggedright\arraybackslash}p{5cm}@{}}
%     \hline
%     \textbf{\textit{Dispositivo}} & \textbf{\textit{Descripción}} \\ 
%     \hline
%     Low-Noise Amplifier  & Amplificador RF banda ancha de alta ganancia (30\,dB), con rango de Operación de 0,1--2000 MHz. \\
%     \hline
%     Mixer - AD831 & Mezclador de un solo chip de amplio rango dinámico y baja distorsión, opera entre 0,1--500\,MHz con 10\,dBm de ganancia, con condensadores de aislamiento DC. \\
%     \hline
%     Bias Tee & RF Microondas HF Bias Tee DC Bias 50K--60\,MHz. Fuente de alimentación de antena activa. \\
%     \hline
%     LED Azul con Fósforo Amarillo & Linterna de luz blanca convencional con una respuesta en frecuencia típica de 2MHz. \\
%     \hline
%     USRP 2900, 2901 & Del fabricante National Instruments con un rango de operación RF entre los 70\,MHz--6\,GHz. \\
%     \hline
%     \end{tabular}
%     \label{Componentes del Transmisor Híbrido.}
% \end{table*}
% %%%%%%%%%%%%%%%%%%%%%%%%%

% %%%%%%%%%%%%%%%%%%%%%%%%%%%%%%%%%%%%%%%%%%%%%%%%%
% \begin{table*}[ht!]
% % \begin{table}[ht!]
% % \begin{table}[H]
%     \caption{\textbf{Componentes del Receptor Híbrido.}}
%     \centering
%     \footnotesize
%     \renewcommand{\arraystretch}{1.3}
%     \begin{tabular}{@{}p{3cm}>{\raggedright\arraybackslash}p{5cm}@{}}
%     \hline
%     \textbf{\textit{Dispositivo}} & \textbf{\textit{Descripción}} \\ 
%     \hline
%     Sensor Óptico ThorLabs & Del fabricante ThorLabs con referencia PDA25K(-EC) y ganancia ajustable.\\
%     \hline
%     Bias Tee & RF Microondas HF Bias Tee DC Bias 50K--60\,MHz. Fuente de alimentación de antena activa.\\
%     \hline
%     Low-Noise Amplifier & Amplificador RF banda ancha de alta ganancia (30\,dB), con rango de Operación de 0,1--2000 MHz.\\
%     \hline
%     Mixer AD831 & Mezclador de un solo chip de amplio rango dinámico y baja distorsión, opera entre 0,1--500\,MHz con 10\,dBm de ganancia, con condensadores de aislamiento DC.\\
%     \hline
%     RTL2832U & El RTL2832U es un chip popular utilizado en dongles de software-defined radio (SDR)\\
%     \hline
%     USRP 2900 & Del fabricante {National Instruments} con un rango de operación RF entre los 70\,MHz--6\,GHz.\\
%     \hline
%     \end{tabular}
%     \label{Componentes del Receptor Híbrido}
% \end{table*}
% %%%%%%%%%%%%%%%%%%%%%%%%%%%%%%%%%%%%%%%%%%%%%%%%%%%%%%%%%%%%%%%%


\clearpage